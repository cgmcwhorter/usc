\documentclass[11pt,letterpaper]{article}
\usepackage[lmargin=1in,rmargin=1in,bmargin=1in,tmargin=1in]{geometry}
\usepackage{style}

\pagenumbering{gobble}


% -------------------
% Content
% -------------------
\begin{document}

% TItle
\begin{center} 
\bfseries
\color{scred}
\LARGE Syllabus Quick Facts \par\vspace{0.2\baselineskip}
\Large Calculus for Business Administration and Social Sciences --- Fall 2024
\end{center} \pspace


% Course Information
\mysection{0.27}{Course Information}{course_info}
\hspace{0.53cm} {\itshape Instructor Email}: \href{mailto:cm264@mailbox.sc.edu}{cm264@mailbox.sc.edu} \par
\hspace{0.53cm} {\itshape Course Webpage}: \href{https://coffeeintotheorems.com/courses/2024-2/fall/math-122/}{https://coffeeintotheorems.com/courses/2024-2/fall/math-122/} \par
\hspace{0.53cm} {\itshape Office Hours}: The instructor's office is LeConte 345C. The office hours are\dots \par \vspace{-0.3cm}
	\begin{table}[!ht]
	\centering
	\begin{tabular}{l || l}
	Mon. & 1:00pm -- 2:00pm \\
	Tues. & 4:00pm -- 5:30pm \\
	Wed. & 1:00pm -- 2:00pm \\
	Thurs. & 4:00pm -- 5:30pm 
	\end{tabular}
	\end{table}


% Grading Components
\mysection{0.27}{Grading Components}{grade_comp}
Course grades are determined by the following components: \par \vspace{-0.3cm}
	\begin{table}[!ht]
        \begin{tabular}{lr}
	Check-Ins & 15\% \\
        Homework & 20\% \\
        Exams & 45\% \\
        Final Exam & 20\% \\
        \end{tabular} 
        \end{table}


% Attendance 
\mysection{0.27}{Attendance}{attendance}
Attend each lecture and show up on time. Anticipated absences should be addressed with the instructor in advance of the absence. If you miss a lecture, you are responsible for any material covered, any work assigned, any course changes made, etc. during the class. Five or more unexcused absences from lectures could result in receiving a grade penalty per additional absence or an `F' in the course. Furthermore, excessive lateness will also count as an absence. \pspace


% Check-Ins 
\mysection{0.27}{Check-Ins}{check}
There will be a check-ins \textit{every} class. Check-ins are meant to be short and simple. These check-ins serve more as a method of gauging whether you are keeping up with the material. It is important that if you are late that you obtain a copy of the check-ins immediately. Check-in solutions will often be discussed following the check-in. Because check-in solutions will often be discussed in class, no make-up check-ins will be given except under extraordinary circumstances determined on a case-by-case basis at the discretion of the instructor. Unless otherwise instructed, there are no calculators, computational devices, notes, or outside assistance of any kind allowed on check-ins. \pspace


% Homeworks 
\mysection{0.27}{Homeworks}{homeworks}
There will typically be a homework assigned each class, due the next class. These homeworks will likely be submitted virtually. Assignments should be started as soon as possible; it is easier to keep up than it is to catch up. You may request extensions on homework assignments (possibly incurring a grade penalty). Requests for extensions should be submitted to the instructor in a timely fashion---do not delay! You are encouraged to work with others on homeworks; however, be sure to carefully abide by the academic integrity standards excepted by the college and instructor. \pspace


% Exams 
\mysection{0.27}{Exams}{exams}
There will be three exams in this course, each worth 15\% of the course grade, for a total of 45\% of the course grade. There will also be a final exam worth 20\% of the course grade. Together, all exams are worth 65\% of the course grade. The final exam grade will be used to replace the lowest of the three exam scores---assuming the final exam score is greater than this lowest score. Furthermore, students achieving an average `raw' exam score of 93 or higher may be exempt from the final exam. No other students may be exempt from the final exam. Each exam covers course material up until the exam preceding it. The final exam is cumulative. There are no make-up exams except under extraordinary circumstances. \pspace


% Course Schedule 
\mysection{0.27}{Course Schedule}{schedule}
The following is a \emph{tentative} schedule for the course and is subject to change. 
        \begin{table}[!ht]
        \hspace{-1cm}
        \scalebox{1}{%
        \begin{tabular}{ll || ll}
        Date & Topic(s) & Date & Topic(s) \\ \hline         
        08/20 & Course Introduction & 10/17 & Fall Break (No Class) \\
        08/22 & Linear Functions (\S1.1--1.2) & 10/22 & Distance \& Accumulated Change (\S5.1) \\
        08/27 & Rates of Change \& Applications (\S1.3--1.4) & 10/24 & Definite Integral (\S5.2--5.3) \\ 
        08/29 & Exponentials/Logarithms (\S1.5--1.7) & 10/29 & Integral Interpretations (\S5.4) \\
        09/03 & Transformations \& Polynomials (\S1.8--1.9) & 10/31 & Fund. Thm. of Calc. (\S5.5, 6.3) \\
        09/05 & Review & 11/05 & Election Day (No Class) \\
        09/10 & Exam 1 & 11/07 & Review \\
        09/12 & Derivatives (\S2.1--2.2) & 11/12 & Exam 3 \\
        09/17 & Derivative Interpretations (\S2.3--2.4) & 11/14 & Integrals (\S6.1--6.2) \\
        09/19 & Marginal Cost/Revenue (\S2.5) & 11/19 & Consumer/Product Surplus (\S6.4) \\
        09/24 & Further Derivatives (\S3.1--3.2) & 11/21 & Consumer/Product Surplus (\S6.4) \\
        09/26 & Derivative Rules (\S3.3--3.4) & 11/26 & Thanksgiving Break (No Class) \\
        10/01 & Max, Mins, \& Inflection (\S4.1--4.2) & 11/28 & Thanksgiving Break (No Class) \\
        10/03 & Max, Mins, \& Inflection (\S4.1--4.2) & 12/03 & Review \\
        10/08 & Review & 12/05 & Review \\
        10/10 & Exam 2 & 12/10 or 12/12 & Final Exam \\
        10/15 & Global Maxes, Cost/Revenue (\S4.3--4.4) & 
        \end{tabular}
        }
        \end{table}


\end{document}