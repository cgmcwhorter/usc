\documentclass[11pt,letterpaper]{article}
\usepackage[lmargin=0.5in,rmargin=0.5in,bmargin=0.5in,tmargin=0.5in]{geometry}
\usepackage{style}

\setlength{\parindent}{0ex}
\pagestyle{empty}

% -------------------
% Content
% -------------------
\begin{document}

\pagenumbering{gobble}

\phantom{.} \hfill {\LARGE \bfseries Bonus Problems} \hfill \phantom{.} 

	\begin{table}[h]
	\centering
	\begin{tabular}{l}
	{\itshape Santa now gone, Krampus comes with a sneer.} \\
	{\itshape ``Forget New Years, tough problems are here!} \\
	{\itshape Panic sets in, and there's no joy in sight.} \\
	{\itshape Face these hard problems, which require great might.}
	\end{tabular}
	\end{table} \par\vspace{0.3cm}

% Problem A.
\textbf{Problem A.} Let $\mathcal{R}$ be the region bounded by the curves $y= 5x^2$ and $y= 5x^3$. Consider the volume given by\dots
	\begin{enumerate}[(a)]
	\item an object whose base is $\mathcal{R}$ and whose cross-sections perpendicular to the $x$-axis are squares. 
	\item an object formed by rotating $\mathcal{R}$ about the $x$-axis.
	\item an object formed by rotating $\mathcal{R}$ about the $y$-axis.
	\end{enumerate}
Choose \textit{\textbf{one}} of (a), (b), or (c) and set-up---\textit{\bfseries but do not evaluate}---an integral which computes the volume of the object in the part you have chosen. Indicate which one you have chosen by circling the part. \par\vspace{1cm}

% Problem B
\textbf{Problem B.} Showing all your work, compute the following\dots
	\[
	\lim_{n \to \infty} \dfrac{1}{n} \sum_{k=0}^{n-1} \sin \left( \frac{\pi k}{n} \right)
	\] \pspace

\end{document}