\documentclass[11pt,letterpaper]{article}
\usepackage[lmargin=0.5in,rmargin=0.5in,bmargin=0.5in,tmargin=0.5in]{geometry}
\usepackage{style}

\setlength{\parindent}{0ex}

% -------------------
% Content
% -------------------
\begin{document}

\pagenumbering{gobble}

\phantom{.} \hfill {\LARGE \bfseries Bonus Problems} \hfill \phantom{.} 

	\begin{table}[h]
	\centering
	\begin{tabular}{l}
	{\itshape Santa now gone, Krampus comes with a sneer.} \\
	{\itshape ``Forget New Years, tough problems are here!} \\
	{\itshape Panic sets in, and there's no joy in sight.} \\
	{\itshape Face these hard problems, which require great might.}
	\end{tabular}
	\end{table} \par\vspace{0.3cm}

% Problem A.
\textbf{Problem A.} Let $\mathcal{R}$ be the region bounded by the curves $y= 5x^2$ and $y= 5x^3$. Consider the volume given by\dots
	\begin{enumerate}[(a)]
	\item an object whose base is $\mathcal{R}$ and whose cross-sections perpendicular to the $x$-axis are squares. 
	\item an object formed by rotating $\mathcal{R}$ about the $x$-axis.
	\item an object formed by rotating $\mathcal{R}$ about the $y$-axis.
	\end{enumerate}
Choose \textit{\textbf{one}} of (a), (b), or (c) and set-up---\textit{\bfseries but do not evaluate}---an integral which computes the volume of the object in the part you have chosen. Indicate which one you have chosen by circling the part. \par\vspace{1cm}

% Problem B
\textbf{Problem B.} Showing all your work, compute the following\dots
	\[
	\lim_{n \to \infty} \dfrac{1}{n} \sum_{k=0}^{n-1} \sin \left( \frac{\pi k}{n} \right)
	\] \pspace

%%%%%%%
% Solutions
%%%%%%%
{\bfseries Solutions.} \pspace

\textbf{Problem A.} \par
	\begin{minipage}[b]{0.2\textwidth}
	\[
	\fbox{
	\begin{tikzpicture}[scale=1,every node/.style={scale=0.5}]
	\begin{axis}[
	grid=both,
	axis lines=middle,
	ticklabel style={fill=blue!5!white},
	xmin= -2.5, xmax=3.5,
	ymin= -0.5, ymax=5.5,
	xtick={-3,-2,...,4},
	ytick={-1,0,...,6},
	minor tick = {-6.5,-5.5,...,6.5},
	xlabel=\(x\),ylabel=\(y\),
	]
	\addplot[name path= F, line width= 0.02cm,samples=100,domain= -2.5:3.5] ({x},{5*x^2});
	\addplot[name path= G, line width= 0.02cm,samples=100,domain= -2.5:3.5] ({x},{5*x^3});
	
	\addplot[color=gray!50] fill between[of= F and G, soft clip={domain= 0:1}];
	\addplot[pattern= north east lines, pattern color=gray!10] fill between[of=F and G, soft clip={domain= 0:1}];
	
	\node at (1.1,1) {\LARGE $y= 5x^3$};
	\node at (0.45,4) {\LARGE $y= 5x^2$};
	\end{axis}
	\end{tikzpicture}
	}
	\]
	\end{minipage} \hfill \begin{minipage}[b]{0.45\textwidth}
		\[
		\begin{gathered}
		5x^2= 5x^3 \\
		0= 5x^3 - 5x^2 \\
		0= 5x^2(x - 1) \\
		\end{gathered}
		\]
	But then either $5x^2= 0$, which implies $x= 0$, or $x - 1= 0$, which implies $x= 1$. Therefore, the curves intersect at $x= 0$ and $x= 1$. If $x= 0$, then $y= 5(0^2)= 0$, i.e. $(0, 0)$, and if $x= 1$, then $y= 5(1^2)= 5$, i.e. $(1, 5)$. Choosing $x= \frac{1}{2}$, we can see that $5(\frac{1}{2})^2= \frac{5}{4}= \frac{10}{8} > 5 (\frac{1}{2})^3= \frac{5}{8}$. Therefore, the curve $y= 5x^2$ is `on top.' Finally, observe that for this region, $x, y \geq 0$. Therefore, $y= 5x^2$ if and only if $x= \frac{y}{5}$, and $y= 5x^3$ if and only if $x= \sqrt[3]{\frac{y}{5}}$. 
	\end{minipage}
	
	\begin{enumerate}[(a)]
	\item 
		\[
		V= \int_0^1 (5x^2 - 5x^3)^2 \; dx
		\]
	
	\item 
		\[
		V= \pi \int_0^1 (5x^2)^2 - (5x^3)^2 \; dx \qquad \text{\itshape OR} \qquad V= 2\pi \int_0^5 y \left( \sqrt[3]{\dfrac{y}{5}} - \sqrt{\dfrac{y}{5}} \right) \;dy
		\]
	
	\item 
		\[
		V= 2\pi \int_0^1 x (5x^2 - 5x^3) \; dx \qquad \text{\itshape OR} \qquad V= \pi \int_0^5 \left( \sqrt[3]{\dfrac{y}{5}} \right)^2 - \left( \sqrt{\dfrac{y}{5}} \right)^2 \; dy
		\]
	\end{enumerate}



\newpage



% Problem B
\textbf{Problem B.} {\itshape First, we rewrite the summation:
	\[
	\lim_{n \to \infty} \dfrac{1}{n} \sum_{k=0}^{n-1} \sin \left( \frac{\pi k}{n} \right)= \lim_{n \to \infty} \dfrac{1}{\pi} \cdot \dfrac{\pi}{n} \sum_{k=0}^{n-1} \sin \left( \frac{\pi k}{n} \right)= \lim_{n \to \infty} \dfrac{1}{\pi} \cdot \dfrac{\pi - 0}{n} \sum_{k=0}^{n-1} \sin \left( \frac{\pi - 0}{n} \, k \right)
	\]
Recognizing $\Delta x:= \frac{\pi - 0}{n}$ as $\Delta x:= \frac{b - a}{n}$ as a step-size, i.e. taking $b= \pi$, $a= 0$, and $n= n$, and given the argument of $\sin(x)$ is $\Delta x \cdot k$ from $k= 0$ to $k= n - 1$, we can recognize this limit as the left-hand Riemann sum with equal widths for $\sin(x)$ from $x= 0$ to $x= \pi$. Therefore, we have\dots
	\[
	\begin{aligned}
	\lim_{n \to \infty} \dfrac{1}{n} \sum_{k=0}^{n-1} \sin \left( \frac{\pi k}{n} \right)&= \lim_{n \to \infty} \dfrac{1}{\pi} \cdot \dfrac{\pi - 0}{n} \sum_{k=0}^{n-1} \sin \left( \frac{\pi - 0}{n} \, k \right) \\[0.3cm]
	&=\dfrac{1}{\pi} \int_0^\pi \sin(x) \;dx \\[0.3cm]
	&= \dfrac{1}{\pi} \cdot -\cos x \bigg|_{x=0}^{x= \pi} \\[0.3cm]
	&= \dfrac{1}{\pi} \cdot \bigg( -\cos \pi - (-\cos 0) \bigg) \\[0.3cm]
	&= \dfrac{1}{\pi} \cdot \bigg(-(-1) - (-1) \bigg) \\[0.3cm]
	&= \dfrac{1}{\pi} \bigg(1 + 1 \bigg) \\[0.3cm]
	&= \dfrac{2}{\pi}
	\end{aligned}
	\]
}

\end{document}