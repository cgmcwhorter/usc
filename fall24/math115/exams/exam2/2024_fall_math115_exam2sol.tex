\documentclass[12pt,letterpaper]{exam}
\usepackage[lmargin=1in,rmargin=1in,tmargin=1in,bmargin=1in]{geometry}
\usepackage{../style/exams}

% -------------------
% Course & Exam Information
% -------------------
\newcommand{\course}{MATH 115: Exam 2}
\newcommand{\term}{Fall --- 2024}
\newcommand{\examdate}{10/28/2024}
\newcommand{\timelimit}{75 Minutes}

\setbool{hideans}{false} % Student: True; Instructor: False

\usepackage{polynom} % Polynomial Division

% -------------------
% Content
% -------------------
\begin{document}

\examtitle
\instructions{Write your name on the appropriate line on the exam cover sheet. This exam contains \numpages\ pages (including this cover page) and \numquestions\ questions. Check that you have every page of the exam. Answer the questions in the spaces provided on the question sheets. Be sure to answer every part of each question and show all your work. If you run out of room for an answer, continue on the back of the page --- being sure to indicate the problem number.} 
\scores
\bottomline
\newpage


% -------------------
% Questions
% -------------------
\begin{questions}

% Question 1
\newpage
\question[10] Complete the table of exact values for the trigonometric figures given below.
	{\def\arraystretch{3}
	\setlength{\tabcolsep}{2.5em}
	\begin{table}[!ht]
	\centering
	\begin{tabular}{|c||c|c|c|c|c|} \hline
	$\theta$ & $0$ & $\dfrac{\pi}{6}$ & $\dfrac{\pi}{4}$ & $\dfrac{\pi}{3}$ & $\dfrac{\pi}{2}$ \\ \hline \hline
	$\sin \theta$ & $0$ & $\dfrac{1}{2}$ & $\dfrac{1}{\sqrt{2}}$ & $\dfrac{\sqrt{3}}{2}$ & $1$ \\ \hline
	$\cos \theta$ & $1$ & $\dfrac{\sqrt{3}}{2}$ & $\dfrac{1}{\sqrt{2}}$ & $\dfrac{1}{2}$ & $0$ \\ \hline
	$\tan \theta$ & $0$ & $\dfrac{1}{\sqrt{3}}$ & $1$ & $\sqrt{3}$ & {\footnotesize undef.} \\ \hline
	\end{tabular}
	\end{table}
	}



% Question 2
\question[8] Compute the following---no work is required:
	\begin{enumerate}[(a)]
	\item $\log_3(27)= 3$ \par\vspace{0.3cm}
	\item $\log_5 \left( \dfrac{1}{25} \right)= -2$ \par\vspace{0.3cm}
	\item $\log_4(2)= \frac{1}{2}$ \par\vspace{0.3cm}
	\item $\log_{\sqrt{2}}(1)= 0$ \par\vspace{0.3cm}
	\item $\ln \left(e^{2/3} \right)= \frac{2}{3}$ \par\vspace{0.3cm}
	\item $\pi^{2 \log_\pi(3)}= \pi^{\log_\pi (3^2)}= 3^2= 9$
	\item $\log_{25} (125)= \frac{\log_5 125}{\log_5 25}= \frac{3}{2}$
	\end{enumerate} \par\vspace{0.2cm}



% Question 3
\question Consider the function $f(x)= 5 \left(3^{2 - x} \right)$.
	\begin{parts}
	\part[5] Write $f(x)$ in the form $Ab^x$ for some $A$ and $b$. \vfill
		\[
		f(x)= 5 \left(3^{2 - x} \right)= 5 \left(3^2 \cdot 3^{-x} \right)= (5 \cdot 9) 3^{-x}= 45 \left( 3^{-1} \right)^x= 45 \left( \dfrac{1}{3} \right)^x
		\] \vfill
	\part[3] Determine whether $f(x)$ is exponentially increasing or decreasing. Use (a) to justify your answer. \vfill
	
	{\itshape Observe that $f(x)$ has the form $Ab^x$ with $A= 45$ and $b= \frac{1}{3}$. Because $A > 0$ and $b= \frac{1}{3}$, we know that $f(x)$ is exponentially decreasing.}
	
	\vfill
	\end{parts}



% Question 4
\newpage
\question[10] Find the quotient and remainder when $2x^3 + x^2 + 8$ is divided by $x + 2$. \pvspace{1cm}

{\itshape \tsol We have\dots
	\[
	\polylongdiv{2x^3 + x^2 + 8}{x + 2}
	\] \pspace
Therefore, the quotient is $2x^2 - 3x + 6$ and the remainder is $-4$.
	\[
	\begin{aligned}
	\text{Quotient: }& 2x^2 - 3x + 6 \\[0.1cm]
	\text{Remainder: }& -4
	\end{aligned}
	\] \pvspace{0.5cm}
Alternatively, because $x + 2$ is linear, we can instead use synthetic division. For synthetic division, we write $x + 2$ in the form $x - a$, i.e. $x + 2= x - (-2)$. Being sure to include the zero terms in $2x^3 + x^2 + 8$, i.e. writing $2x^3 + x^2 + 8= 2x^3 + x^2 + 0x + 8$, we have\dots
	\[
	\polyhornerscheme[x=-2]{2x^3 + x^2 + 8}
	\]
Therefore, we can see that the quotient is $2x^2 - 3x + 6$ and the remainder is $-4$.
	\[
	\begin{aligned}
	\text{Quotient: }& 2x^2 - 3x + 6 \\[0.1cm]
	\text{Remainder: }& -4
	\end{aligned}
	\] 
}



% Question 5
\newpage
\question[10] Showing all your work, solve the following: \par\vspace{0.5cm}
	\begin{parts}
	\part $x(2x - 3)= 5$ \pvspace{0.2cm}
		\[
		\begin{gathered}
		x(2x - 3)= 5 \\[0.3cm]
		2x^2 - 3x= 5 \\[0.3cm]
		2x^2 - 3x - 5= 0 \\[0.3cm]
		(2x - 5)(x + 1)= 0 \\[0.3cm]
		2x - 5= 0 \quad \text{ or } \quad x + 1= 0 \\[0.3cm]
		x= \frac{5}{2} \quad \text{ or } \quad x= -1
		\end{gathered}
		\] \pvspace{4.45cm}
	
	\part $\dfrac{x}{x^2 - 4}= \dfrac{2}{x + 2}$ \pspace
		\[
		\begin{gathered}
		\dfrac{x}{x^2 - 4}= \dfrac{2}{x + 2} \\[0.3cm]
		x(x + 2)= 2(x^2 - 4) \\[0.3cm]
		x^2 + 2x= 2x^2 - 8 \\[0.3cm]
		0= x^2 - 2x - 8 \\[0.3cm]
		0= (x - 4)(x + 2) \\[0.3cm]
		x - 4= 0 \quad \text{ or } \quad x + 2= 0 \\[0.3cm]
		x= 4 \quad \text{ or } \quad x= -2 \\[0.3cm]
		x \neq -2 \text{ \itshape (Extraneous) } \Longrightarrow x= 4
		\end{gathered} \hspace{1cm}
		\text{ \itshape OR } \hspace{1cm}
		\begin{gathered}
		\dfrac{x}{x^2 - 4}= \dfrac{2}{x + 2} \\[0.3cm]
		\dfrac{x}{(x - 2)(x + 2)}= \dfrac{2}{x + 2} \\[0.3cm]
		\dfrac{x}{x - 2}= 2 \\[0.3cm]
		x= 2(x - 2) \\[0.3cm]
		x= 2x - 4 \\[0.3cm]
		-x= -4 \\[0.3cm]
		x= 4
		\end{gathered}
		\]
	\end{parts}



% Question 6
\newpage
\question[10] Showing all your work, solve the following: \par\vspace{0.5cm}
	\begin{parts}
	\part $|2x - 5|= 7$ \pspace
	
	{\itshape Because $|x|= x$ or $|x|= -x$, depending on the value of $x$, if $|2x - 5|= 7$, then either $2x - 5=7$ or $-(2x - 5)= 7$. But then\dots \pvspace{1cm}
		\[
		\begin{gathered}
		2x - 5= 7 \\[0.3cm]
		2x= 12 \\[0.3cm]
		x= 6
		\end{gathered}
		\qquad\qquad \text{ OR } \qquad\qquad
		\begin{gathered}
		-(2x - 5)= 7 \\[0.3cm]
		-2x + 5= 7 \\[0.3cm]
		-2x= 2 \\[0.3cm]
		x= -1
		\end{gathered}
		\]
	} \pvspace{3.7cm}
	
	\part $3 - x \geq 4x - 7$ \pvspace{1cm}
		\[
		\begin{gathered}
		3 - x \geq 4x - 7 \\[0.3cm]
		3 \geq 5x - 7 \\[0.3cm]
		10 \geq 5x \\[0.3cm]
		2 \geq x
		\end{gathered}
		\qquad\qquad \text{ OR } \qquad\qquad
		\begin{gathered}
		3 - x \geq 4x - 7 \\[0.3cm]
		3 - 5x \geq -7 \\[0.3cm]
		-5x \geq -10 \\[0.3cm]
		x \leq 2
		\end{gathered}
		\] 
	\end{parts}



% Question 7
\newpage
\question[8] Showing all your work, write the following as a single logarithm:
	\[
	3\log_2(x) - \frac{1}{4} \log_2(y) + 5
	\] \pvspace{0.5cm}
	
	\[
	\begin{gathered}
	3\log_2(x) - \frac{1}{4} \log_2(y) + 5 \\[0.3cm]
	\log_2(x^3) - \log_2(y^{1/4}) + \log_2(2^5) \\[0.3cm]
	\log_2 \left( \dfrac{2^5 x^3}{y^{1/4}} \right) \\[0.3cm]
	\log_2 \left( \dfrac{32x^3}{\sqrt[4]{y}} \right)
	\end{gathered}
	\] \pvspace{2.1cm}



% Question 8
\question[8] Using the fact that $\log_b(x)= -3$, $\log_b(y)= 4$, and $\log_b(z)= 1$, find the exact value of the expression below---be sure to show all your work.
	\[
	\log_b \left( \dfrac{x \sqrt{y}}{z} \right)
	\] \pvspace{0.5cm}
	
	\[
	\begin{gathered}
	\log_b \left( \dfrac{x \sqrt{y}}{z} \right) \\[0.3cm]
	\log_b \left( x \sqrt{y} \right) - \log_b(z) \\[0.3cm]
	\log_b (x) + \log_b (\sqrt{y}) - \log_b(z) \\[0.3cm]
	\log_b (x) + \log_b (y^{1/2}) - \log_b(z) \\[0.3cm]
	\log_b (x) + \frac{1}{2}\, \log_b (y) - \log_b(z) \\[0.3cm]
	-3 + \frac{1}{2} \cdot 4 - 1 \\[0.3cm]
	-3 + 2 - 1 \\[0.3cm]
	-2
	\end{gathered}
	\]



% Question 9
\newpage
\question[10] Showing all your work, solve the following equations: \par\vspace{0.5cm}
	\begin{parts}
	\part $3e^{-5x} - 11= 19$ \vfill
		\[
		\begin{gathered}
		3e^{-5x} - 11= 19 \\[0.3cm]
		3e^{-5x}= 30 \\[0.3cm]
		e^{-5x}= 10 \\[0.3cm]
		\ln(e^{-5x})= \ln(10) \\[0.3cm]
		-5x= \ln(10) \\[0.3cm]
		x= -\dfrac{\ln(10)}{5} \approx -0.460517
		\end{gathered}
		\] \vfill \vspace{3cm}
	\part $2 \log_{11} x - \log_{11} 4 = \log_{11} 25$ \vfill
		\[
		\begin{gathered}
		2 \log_{11} x - \log_{11} 4 = \log_{11} 25 \\[0.3cm]
		\log_{11} x^2 - \log_{11} 4 = \log_{11} 25 \\[0.3cm]
		\log_{11} \left( \dfrac{x^2}{4} \right) = \log_{11} 25 \\[0.3cm]
		11^{\log_{11} \left( \frac{x^2}{4} \right)}= 11^{\log_{11} 25} \\[0.3cm]
		\dfrac{x^2}{4}= 25 \\[0.3cm]
		x^2= 100 \\[0.3cm]
		x= \pm 10 \\[0.3cm]
		\text{Because the domain of} \\
		\text{$\log_{11} x$ is $x > 0$, we have\dots} \\[0.3cm]
		x= 10
		\end{gathered}
		\qquad \qquad \text{ OR } \qquad \qquad 
		\begin{gathered}
		2 \log_{11} x - \log_{11} 4 = \log_{11} 25 \\[0.3cm]
		2 \log_{11} x= \log_{11} 4 + \log_{11} 25 \\[0.3cm]
		2 \log_{11} x= \log_{11} (100) \\[0.3cm] 
		\log_{11} x= \dfrac{1}{2}\, \log_{11} (100) \\[0.3cm] 
		\log_{11} x= \log_{11} 10 \\[0.3cm]
		11^{\log_{11} x}= 11^{\log_{11} 10} \\[0.3cm] 
		x= 10
		\end{gathered}
		\]
	\end{parts}



% Question 10
\newpage
\question[10] Showing all your work, use the quadratic formula to solve the following:	
	\[
	3x^2= 1 - 2x
	\] \pvspace{1cm}

{\itshape \tsol We have\dots
	\[
	\begin{gathered}
	3x^2= 1 - 2x \\[0.3cm]
	3x^2 + 2x - 1= 0 
	\end{gathered}
	\] \pspace
Therefore, we have $a= 3$, $b= 2$, and $c= -1$. But then\dots \pspace
	\[
	\begin{aligned}
	x&= \dfrac{-b \pm \sqrt{b^2 - 4ac}}{2a} \\[0.3cm]
	&= \dfrac{-2 \pm \sqrt{2^2 - 4(3)(-1)}}{2(3)} \\[0.3cm]
	&= \dfrac{-2 \pm \sqrt{4 + 12}}{6} \\[0.3cm]
	&= \dfrac{-2 \pm \sqrt{16}}{6} \\[0.3cm]
	&= \dfrac{-2 \pm 4}{6} 
	\end{aligned}
	\] \pspace
Therefore, either $x= \dfrac{-2 - 4}{6}= \dfrac{-6}{6}= -1$ or $x= \dfrac{-2 + 4}{6}= \dfrac{2}{6}= \dfrac{1}{3}$.
}



% Question 11
\newpage
\question[8] Consider the following function:
	\[
	g(x)= \dfrac{(2x + 5)(x - 4)}{(x + 2)(x - 3)}
	\] \par\vspace{0.25cm}

\begin{enumerate}[(a)]
\item Determine the domain of $g(x)$. \vfill

{\itshape The domain of a `reduced' rational function $\frac{f(x)}{g(x)}$ is the $x$-values for which $g(x) \neq 0$. We see that $g(x)$ is reduced and $(x + 2)(x - 3)= 0$ if and only if $x= -2, 3$. Therefore, the domain is the set of all reals such that $x \neq -2, 3$, i.e. $(-\infty, -2) \cup (-2, 3) \cup (3, \infty)$.} \vfill

\item Find the $x$-intercepts for $g(x)$. \vfill

{\itshape The $x$-intercepts of a `reduced' rational function $\frac{f(x)}{g(x)}$ are the $x$-values such that $f(x)= 0$. Because $g(x)$ is `reduced' and $(2x + 5)(x - 4)= 0$ if and only if $x= -\frac{5}{2}, 4$, the $x$-intercepts of $g(x)$ are $x= -\frac{5}{2}, 4$.} \vfill

\item Find any vertical asymptotes for $g(x)$. \vfill

{\itshape The vertical asymptotes of a `reduced' rational function $\frac{f(x)}{g(x)}$ are the $x$-values for which $g(x)= 0$. We see that $g(x)$ is reduced and $(x + 2)(x - 3)= 0$ if and only if $x= -2, 3$. Therefore, the vertical asymptotes of $g(x)$ are $x= -2$ and $x= 3$.} \vfill

\item Find any horizontal asymptotes for $g(x)$. \vfill

{\itshape We know the horizontal asymptote (if any) of a rational function $\frac{f(x)}{g(x)}$ depends only on the degrees of $f(x)$ and $g(x)$. The numerator and denominator of $g(x)$ have degree two. Therefore, the horizontal asymptote is the ratio of the leading coefficients. We have $g(x)= \frac{(2x + 5)(x - 4)}{(x + 2)(x - 3)}= \frac{2x^2 - 3x - 20}{x^2 - x - 6}$. Therefore, the horizontal asymptotes is $y= 2$.} \vfill
\end{enumerate}


\end{questions}
\end{document}