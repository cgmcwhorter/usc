\documentclass[12pt,letterpaper]{exam}
\usepackage[lmargin=1in,rmargin=1in,tmargin=1in,bmargin=1in]{geometry}
\usepackage{../style/exams}

% -------------------
% Course & Exam Information
% -------------------
\newcommand{\course}{MATH 115: Exam 2}
\newcommand{\term}{Fall --- 2024}
\newcommand{\examdate}{10/28/2024}
\newcommand{\timelimit}{75 Minutes}

\setbool{hideans}{true} % Student: True; Instructor: False


% -------------------
% Content
% -------------------
\begin{document}

\examtitle
\instructions{Write your name on the appropriate line on the exam cover sheet. This exam contains \numpages\ pages (including this cover page) and \numquestions\ questions. Check that you have every page of the exam. Answer the questions in the spaces provided on the question sheets. Be sure to answer every part of each question and show all your work. If you run out of room for an answer, continue on the back of the page --- being sure to indicate the problem number.} 
\scores
\bottomline
\newpage


% -------------------
% Questions
% -------------------
\begin{questions}

% Question 1
\newpage
\question[10] Complete the table of exact values for the trigonometric figures given below.
	{\def\arraystretch{3}
	\setlength{\tabcolsep}{2.5em}
	\begin{table}[!ht]
	\centering
	\begin{tabular}{|c||c|c|c|c|c|} \hline
	$\theta$ & $0$ & $\dfrac{\pi}{6}$ & $\dfrac{\pi}{4}$ & $\dfrac{\pi}{3}$ & $\dfrac{\pi}{2}$ \\ \hline \hline
	$\sin \theta$ &  &  &  &  &  \\ \hline
	$\cos \theta$ &  &  &  &  &  \\ \hline
	$\tan \theta$ &  &  &  &  &  \\ \hline
	\end{tabular}
	\end{table}
	}



% Question 2
\question[8] Compute the following---no work is required:
	\begin{enumerate}[(a)]
	\item $\log_3(27)=$ \par\vspace{0.3cm}
	\item $\log_5 \left( \dfrac{1}{25} \right)=$ \par\vspace{0.3cm}
	\item $\log_4(2)=$ \par\vspace{0.3cm}
	\item $\log_{\sqrt{2}}(1)=$ \par\vspace{0.3cm}
	\item $\ln \left(e^{2/3} \right)=$ \par\vspace{0.3cm}
	\item $\pi^{2 \log_\pi(3)}=$
	\item $\log_{25} (125)=$
	\end{enumerate} \par\vspace{0.2cm}



% Question 3
\question Consider the function $f(x)= 5 \left(3^{2 - x} \right)$.
	\begin{parts}
	\part[5] Write $f(x)$ in the form $Ab^x$ for some $A$ and $b$. \par\vspace{2.5cm}
	\part[3] Determine whether $f(x)$ is exponentially increasing or decreasing. Use (a) to justify your answer. 
	\end{parts}



% Question 4
\newpage
\question[10] Find the quotient and remainder when $2x^3 + x^2 + 8$ is divided by $x + 2$. 



% Question 5
\newpage
\question[10] Showing all your work, solve the following: \par\vspace{0.5cm}
	\begin{parts}
	\part $x(2x - 3)= 5$ \vfill
	\part $\dfrac{x}{x^2 - 4}= \dfrac{2}{x + 2}$ \vfill
	\end{parts}



% Question 6
\newpage
\question[10] Showing all your work, solve the following: \par\vspace{0.5cm}
	\begin{parts}
	\part $|2x - 5|= 7$ \vfill
	\part $3 - x \geq 4x - 7$ \vfill
	\end{parts}



% Question 7
\newpage
\question[8] Showing all your work, write the following as a single logarithm:
	\[
	3\log_2(x) - \frac{1}{4} \log_2(y) + 5
	\] \par\vspace{8cm}



% Question 8
\question[8] Using the fact that $\log_b(x)= -3$, $\log_b(y)= 4$, and $\log_b(z)= 1$, find the exact value of the expression below---be sure to show all your work.
	\[
	\log_b \left( \dfrac{x \sqrt{y}}{z} \right)
	\]
	


% Question 9
\newpage
\question[10] Showing all your work, solve the following equations: \par\vspace{0.5cm}
	\begin{parts}
	\part $3e^{-5x} - 11= 19$ \vfill
	\part $2 \log_{11} x - \log_{11} 4 = \log_{11} 25$ \vfill
	\end{parts}



% Question 10
\newpage
\question[10] Showing all your work, use the quadratic formula to solve the following:	
	\[
	3x^2= 1 - 2x
	\]



% Question 11
\newpage
\question[8] Consider the following function:
	\[
	g(x)= \dfrac{(2x + 5)(x - 4)}{(x + 2)(x - 3)}
	\] \par\vspace{0.25cm}

\begin{enumerate}[(a)]
\item Determine the domain of $g(x)$. \vfill
\item Find the $x$-intercepts for $g(x)$. \vfill
\item Find any vertical asymptotes for $g(x)$. \vfill
\item Find any horizontal asymptotes for $g(x)$. \vfill
\end{enumerate}


\end{questions}
\end{document}