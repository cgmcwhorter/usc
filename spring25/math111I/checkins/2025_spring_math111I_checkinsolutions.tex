\documentclass[11pt,letterpaper]{article}
\usepackage[lmargin=1in,rmargin=1in,bmargin=1in,tmargin=1in]{geometry}
\usepackage{checkins}


% -------------------
% Content
% -------------------
\begin{document}
\thispagestyle{title}

% 01/15
\checkin{01/15} The relation given by $f(x)= \dfrac{x + 6}{3x - 5}$ is a function. \pspace

\sol The statement is \textit{true}. For each input $x$, there is only one possible output---namely, the one obtained by plugging in the given $x$-value and following order of operations. For instance, given $x= 2$, we have\dots
	\[
	f(2)= \dfrac{2 + 6}{3(2) - 5}= \dfrac{8}{6 - 5}= \dfrac{8}{1}= 8
	\] \pvspace{1.3cm}



% 01/16
\checkin{01/16} Suppose Didly is trying to predict the amount of prison years that he will receive for the number of crimes that he has (allegedly) committed. The number of years would be the independent variable, the number of crimes would be the dependent variable, and the number of years is a function of the number of crimes. \pspace

\sol The statement is \textit{false}. Didly is trying to predict the number of years in prison. This must be the output of his `model', i.e. function. Therefore, it must be that the number of years in prison is the dependent variable. Because he is trying to make the prediction using the number of crimes, i.e. use this as the input to his model, the number of crimes must be the independent variable. However, it is true that the number of years should be a function of the number of crimes because Didly can only be sentenced to one possible number of years given the number of crimes that he has committed. \pvspace{1.3cm}




% 01/21
\checkin{01/21} The equation $\dfrac{2x - 6}{2}$ is equivalent to $x - 6$. \pspace

\sol The statement is \textit{false}. First, $\frac{2x - 6}{2}$ is not an equation---nor is $x - 6$ an equation. The statement should read, ``The expression $\dfrac{2x - 6}{2}$ is equivalent to $x - 6$.'' However, these expressions still are not equivalent. If the expressions are equivalent, then they yield the same value for \textit{every} valid $x$-value. If we substitute $x= 1$ into each expression, we have\dots
	\[
	\begin{aligned}
	\dfrac{2x - 6}{2} \, \bigg|_{x= 1}&= \dfrac{2(1) - 6}{2}= \dfrac{2 - 6}{2}= \dfrac{-4}{2}= -2 \\[0.2cm]
	x - 6 \, \bigg|_{x= 1}&= 1 - 6= -5
	\end{aligned}
	\]
Because these are not the same, these expressions cannot be equivalent. \pvspace{1.3cm}



%% 01/22
%\checkin{01/22} If the line $y= a$ intersects the graph of $f(x)$, then there \textit{must} be a solution to the equation $f(x)= a$. \pspace
%
%\sol The statement is \textit{true}. If the graph of $f(x)$ and the line $y= a$ intersect, then there is a point $(x_0, y_0)$ on the graph of both. But each point on the graph of $f(x)$ has the property that $y= f(x)$ so that $y_0= f(x_0)$. But because this point also lies along the line $y= a$, we know that $y_0= a$. Then $f(x_0)= y_0= a$, i.e. $f(x_0)= a$. Therefore, there is at least one $x$-value such that $f(x)= a$---namely, $x= x_0$. \pvspace{1.3cm}



%% 01/23
%\checkin{01/23} If $(-1, 3)$ and $(5, -2)$ are points on the graph of $f(x)$, then the average rate of change for $f(x)$ on $[-1, 5]$ is $-\frac{2}{3}$. \pspace
%
%\sol The statement is \textit{false}. The average rate of change for a function $f(x)$ on an interval $[a, b]$ is $\frac{f(b) - f(a)}{b - a}$, i.e. the slope of the line segment through the points $(a, f(a))$ and $(b, f(b))$. But then the average rate of change for this function is\dots
%	\[
%	\text{Avg. ROC}= \dfrac{3 - (-2)}{-1 - 5}= \dfrac{5}{-6}= -\dfrac{5}{6}
%	\]









\end{document}