\documentclass[11pt,letterpaper]{article}
\usepackage[lmargin=1in,rmargin=1in,bmargin=1in,tmargin=1in]{geometry}
\usepackage{checkins}


% -------------------
% Content
% -------------------
\begin{document}
\thispagestyle{title}

% 01/15
\checkin{01/15} The relation given by $f(x)= \dfrac{x + 6}{3x - 5}$ is a function. \pspace

\sol The statement is \textit{true}. For each input $x$, there is only one possible output---namely, the one obtained by plugging in the given $x$-value and following order of operations. For instance, given $x= 2$, we have\dots
	\[
	f(2)= \dfrac{2 + 6}{3(2) - 5}= \dfrac{8}{6 - 5}= \dfrac{8}{1}= 8
	\] \pvspace{1.3cm}



% 01/16
\checkin{01/16} Suppose Didly is trying to predict the amount of prison years that he will receive for the number of crimes that he has (allegedly) committed. The number of years would be the independent variable, the number of crimes would be the dependent variable, and the number of years is a function of the number of crimes. \pspace

\sol The statement is \textit{false}. Didly is trying to predict the number of years in prison. This must be the output of his `model', i.e. function. Therefore, it must be that the number of years in prison is the dependent variable. Because he is trying to make the prediction using the number of crimes, i.e. use this as the input to his model, the number of crimes must be the independent variable. However, it is true that the number of years should be a function of the number of crimes because Didly can only be sentenced to one possible number of years given the number of crimes that he has committed. \pvspace{1.3cm}




% 01/21
\checkin{01/21} The equation $\dfrac{2x - 6}{2}$ is equivalent to $x - 6$. \pspace

\sol The statement is \textit{false}. First, $\frac{2x - 6}{2}$ is not an equation---nor is $x - 6$ an equation. The statement should read, ``The expression $\dfrac{2x - 6}{2}$ is equivalent to $x - 6$.'' However, these expressions still are not equivalent. If the expressions are equivalent, then they yield the same value for \textit{every} valid $x$-value. If we substitute $x= 1$ into each expression, we have\dots
	\[
	\begin{aligned}
	\dfrac{2x - 6}{2} \, \bigg|_{x= 1}&= \dfrac{2(1) - 6}{2}= \dfrac{2 - 6}{2}= \dfrac{-4}{2}= -2 \\[0.2cm]
	x - 6 \, \bigg|_{x= 1}&= 1 - 6= -5
	\end{aligned}
	\]
Because these are not the same, these expressions cannot be equivalent. \pvspace{1.3cm}



% 01/22
\checkin{01/22} If the line $y= a$ intersects the graph of $f(x)$, then there \textit{must} be a solution to the equation $f(x)= a$. \pspace

\sol The statement is \textit{true}. If the graph of $f(x)$ and the line $y= a$ intersect, then there is a point $(x_0, y_0)$ on the graph of both. But each point on the graph of $f(x)$ has the property that $y= f(x)$ so that $y_0= f(x_0)$. But because this point also lies along the line $y= a$, we know that $y_0= a$. Then $f(x_0)= y_0= a$, i.e. $f(x_0)= a$. Therefore, there is at least one $x$-value such that $f(x)= a$---namely, $x= x_0$. \pvspace{1.3cm}



% 01/23
\checkin{01/23} If $(-1, 3)$ and $(5, -2)$ are points on the graph of $f(x)$, then the average rate of change for $f(x)$ on $[-1, 5]$ is $-\frac{2}{3}$. \pspace

\sol The statement is \textit{false}. The average rate of change for a function $f(x)$ on an interval $[a, b]$ is $\frac{f(b) - f(a)}{b - a}$, i.e. the slope of the line segment through the points $(a, f(a))$ and $(b, f(b))$. But then the average rate of change for this function is\dots
	\[
	\text{Avg. ROC}= \dfrac{3 - (-2)}{-1 - 5}= \dfrac{5}{-6}= -\dfrac{5}{6}
	\] \pvspace{1.3cm}



% 01/27
\checkin{01/28} Functions that have a constant rate of change must be linear. \pspace

\sol The statement is \textit{true}. A function being linear is equivalent to any of the following: it has a constant rate of change, it can be expressed as $y= mx + b$ for some $m, b$, and its graph is a line. To see why a function with a constant rate of change must be linear, suppose the value at $x= 0$ is $5$. Suppose the constant rate of change for each increase of 1 in $x$ is $-4$, i.e. every increase of 1 in $x$ results in a decrease of 4 in the output. At $x= 1$, we know $y= 5 - 4= 1$. At $x= 2$, we know that $y= 5 - 4 - 4= 5 - 4(2)= -3$. At $x= 3$, we know that $y= 5 - 4 - 4 - 4= 5 - 4(3)= -7$. Generally, given $x$, we can see that $y= 5 - 4x$. Letting $m= -4$ and $b= 5$, we can see that $y= mx + b$. But nothing is special about the choice of $5$ for the `starting' value or $m= -4$ for the rate of change---this would merely change the numbers. Therefore, any linear function should be expressible as $y= mx + b$. \pvspace{1.3cm}



% 01/28
\checkin{01/28} If $\ell(x)= \dfrac{6x - 12}{-2}$, then $\ell$ is linear with slope $-3$ and $y$-intercept $6$. \pspace

\sol The statement is \textit{true}. Observe that $\ell(x)= \dfrac{6x - 12}{-2}= \dfrac{6x}{-2} - \dfrac{12}{-2}= -3x - (-6)= -3x + 6$. Observe that $\ell(x)$ has the form $y= mx + b$ with $y= \ell$, $m= -3$, $x= x$, and $b= 6$. Because $\ell(x)$ has the form $y= mx + b$, $\ell(x)$ must be linear. We know also that the slope is $m= -3$ and the $y$-intercept is $b= 6$. \pvspace{1.3cm}



% 01/29
\checkin{01/29} If a line has slope $8$ and contains the point $(1, 5)$, then the equation of the line is $y= 5 + 8(x + 1)$. \pspace

\sol The statement is \textit{false}. There are many ways to determine this. For instance, recall the point-slope form of a line: if a line has slope $m$ and contains a point $(x_0, y_0)$, then the equation for the line is $y= y_0 + m(x - x_0)$. We know the line has slope 8 and contains $(1, 5)$; therefore, the equation of the line must be $y= y_0 + m(x - x_0)= 5 + 8(x - 1)$, which is not the given line. Therefore, the statement is false. Alternatively, if a line contains a point, that point must satisfy the equation of the line. So, if $(1, 5)$ is on the given line, then $y= 5$ when $x= 1$. However, when $x= 1$, we can see that $y= 5 + 8(1 + 1)= 5 + 8(2)= 5 + 16= 21$, which is not 5. Therefore, the statement is false. \pvspace{1.3cm}



% 01/30
\checkin{01/30} The line $y= 5 - 3(x + 2)$ has slope $-3$ and contains the point $(-2, 5)$. \pspace

\sol The statement is \textit{true}. There are many ways to determine this. For instance, recall the point-slope form of a line: if a line has slope $m$ and contains a point $(x_0, y_0)$, then the equation for the line is $y= y_0 + m(x - x_0)$. We know that if the line had slope $-3$ and contained the point $(-2, 5)$, the equation of the line would be $y= y_0 + m(x - x_0)= 5 + (-8) \big(x - (-2) \big)= 5 - 8(x + 2)$, which is the given line. Therefore, the statement is true. Alternatively, we have $y= 5 - 3(x + 2)= 5 - 3x - 6= -3x - 1$. We can see that the line has slope $m= -3$. If the line contains the point $(-2, 5)$, then it satisfies the equation of the line. So, if $x= -2$, we know that $y= 5$. Observe that when $x= -2$, we have $y= -3(-2) - 1= 6 - 1= 5$. Therefore, the statement is true. \pvspace{1.3cm}



% 02/03
\checkin{02/03} The average rate of change for a linear function is its slope. \pspace

\sol The statement is \textit{true}. We know that the rate of change of a linear function is its slope, $m$. This rate of change is constant because the function is linear. But then computing the average rate of change for a linear function must always be the same---the value of $m$. \pvspace{1.3cm}



% 02/04
\checkin{02/04} It is possible for lines to be both parallel and perpendicular. \pspace

\sol The statement is \textit{false}. This is best thought about geometrically. Although we know that two linear functions are parallel if and only if their slopes are equal, geometrically, this means that the lines never intersect. Furthermore, while we know two linear functions are perpendicular if and only if their slopes are negative reciprocals, geometrically, this means that the lines intersect at right angles. But then if lines are parallel, they cannot intersect. But if they are perpendicular, they must intersect (at right angles). So, lines cannot be both parallel and perpendicular. \pvspace{1.3cm}



% 02/05
\checkin{02/05} If Carroll and John each begin with different amounts of Cesium-137 that decay linearly at equal rates, then Carroll and John will never have the same amount of Cesium-137. \pspace

\sol The statement is \textit{true}. We are told that the amount of Cesium-137 both Carroll and John have is given by a linear function (because we are told they decay linearly). But because the Cesium-137 for both individuals decays at the same rate, the slopes (rates of change) for these linear functions must be equal. But then if the lines are distinct, they must be parallel. But we know Carroll and John start with different amounts of Cesium-137; therefore, the amount of Cesium-137 that the linear models say Carroll and John have will never be the same. \pvspace{1.3cm}



% 02/06
\checkin{02/06} A system of two linear equations consists of non-parallel lines, then the system has a solution. \pspace

\sol The statement is \textit{true}. We know a solution to a system of equations corresponds to point(s) that lie on the graphs given by each equation. Because the lines are \textit{not} parallel, they must intersect. But this intersection point corresponds to a solution to the system of equations. \pvspace{1.3cm}



% 02/10
\checkin{02/10} It is possible for a system of linear equations to have fifteen solutions. \pspace

\sol The statement is \textit{false}. If a system of linear equations is given by lines which are parallel, there are no solutions to the system of equations, i.e. there are no solutions. If a system of linear equations are given by the same line, then \textit{every} point along the line is a solution to the system of equations, i.e. there are infinitely many solutions. If the lines are distinct and non-parallel, then the lines intersect at a single point. This point of intersection is the solution to the system of equations, i.e. there is only one solution. But then a system of linear equations either has none, one, or infinitely many solutions. It is then impossible for there to be any other number of solutions. \pvspace{1.3cm}



% 02/11
\checkin{02/11} There is a solution to\dots
	\[
	\begin{cases}
	5x - y= 10 \\
	-10x + 2y= -20
	\end{cases}
	\] \pspace

\sol The statement is \textit{true}. Observe that taking the first equation and multiplying by $-2$, we have $-10x + 2y= -20$. Equivalently, dividing the entire second equation by $-2$, we have $5x - y= 10$. That is, the two equations are the same, i.e. the lines are the same. But then they share \textit{every} point in common. But then any point along this line is a solution. The line given by $5x - y= 10$ is $y= 5x - 10$. Therefore, given any $x$ such that $y= 5x - 10$ is a solution to the system of equations, e.g. $(0, -10), (5, 15), (-3, -25)$, etc. There is then not only a solution but infinitely many solutions. \pvspace{1.3cm}



% 02/12
\checkin{02/12} To find the average rate of change for $f(x)$ on $[1, 7]$, one needs to compute $\dfrac{f(1) - f(7)}{1 - 7}$. \pspace

\sol The statement is \textit{true}. Recall that the average rate of change for a function on an interval is the slope of the line through the endpoints of the function on the interval. That is, the average rate of change for $f(x)$ on $[a, b]$ is the slope of the line through $\big(a, f(a) \big)$ and $\big(b, f(b) \big)$. Therefore, the average rate of change for $f(x)$ on $[1, 7]$ is\dots
	\[
	\dfrac{f(7) - f(1)}{7 - 1}= \dfrac{f(1) - f(7)}{1 - 7}
	\]



\newpage



% 02/18
\checkin{02/18} The function $y= 2(x + 1)^2 - 3$ is a quadratic function with $a= 4$, $b= 4$, and $c= -3$. \pspace

\sol The statement is \textit{false}. Observe that\dots
	\[
	y= 2(x + 1)^2 - 3= 2(x + 1)(x + 1) - 3= 2(x^2 + 2x + 1) - 3= 2x^2 + 4x + 2 - 3= 2x^2 + 4x - 1
	\]
Because $y$ has the form $ax^2 + bx + c$, where $a= 2$, $b= 4$, and $c= -1$, the function is quadratic. However, the given $a, b, c$ values are incorrect. \pvspace{1.3cm}



% 02/19
\checkin{02/19} The quadratic function $y= 5 - (x + 2)^2$ has vertex $(-2, 5)$ and opens upwards. This implies that $y$ has a minimum value. \pspace

\sol The statement is \textit{false}. The vertex form of a quadratic function is $f(x)= a(x - P)^2 + Q$, where $(P, Q)$ is the location of the vertex. We have $y= 5 - (x + 2)^2= -(x + 2)^2 + 5= -\big(x - (-2) \big)^2 + 5$ so that $a= -1$, $P= -2$, and $Q= 5$. Therefore, the vertex of the quadratic function is the point $(-2, 5)$. Generally, suppose a quadratic function is written in the form $y= a(x \pm p)^2 + q$. The $y$-coordinate of the vertex is the constant term and the $x$-coordinate of the vertex is the $x$-value for which the quadratic term vanishes. So for $y= 5 - (x + 2)^2$, the constant term is 5 and the $x$-value that makes the quadratic term vanish, i.e. equal 0, is $x= -2$. Therefore, the vertex is $(-2, 5)$. However, we know that if $a < 0$, the quadratic function opens upwards, whereas if $a > 0$, the quadratic function opens upwards. Because $a= -1 < 0$, this quadratic function opens downwards. This would imply that $y$ has a maximum value. [If the quadratic function opened upwards, then there would be a minimum $y$-value.] \pvspace{1.3cm}



% 02/20
\checkin{02/20} The function $f(x)= 2(x + 3)^2 - 12$ has a minimum value, which is $12$, and occurs when $x= 3$. \pspace

\sol The statement is \textit{false}. The vertex form of a quadratic function is $f(x)= a(x - P)^2 + Q$, where $(P, Q)$ is the location of the vertex. We have $f(x)= 2(x + 3)^2 - 12= 2 \big(x - (-3) \big)^2 - 12$ so that $a= 2$, $P= -3$, and $Q= -12$. Because $a= 2 > 0$, we know that this quadratic function opens upwards, i.e. is convex. This shows that there is a minimum $y$-value. We know this minimum value is the constant term in the vertex form, which here is $-12$, and occurs at the $x$-coordinate of the vertex, which here is $x= -3$. Alternatively, because $a= 2 > 0$, this quadratic function opens upwards. This shows that there is a minimum $y$-value. Because the function is in vertex form, this minimum value occurs when the `quadratic term' is 0, which occurs when $x= -3$, and then is the remaining value once the `quadratic term' has vanished, which here is $-12$. \pvspace{1.3cm}



% 02/24
\checkin{02/24} The quadratic formula states that the solutions to $ax^2 + bx + c= 0$ are $x= \dfrac{-b \pm \sqrt{D}}{a}$, where $D$ is the discriminant of $ax^2 + bx + c$, and is derived from completing the square. \pspace

\sol The statement is \textit{false}. The quadratic formula is $x= \dfrac{-b \pm \sqrt{D}}{2a}$, where $D$ is the discriminant of $ax^2 + bx + c$. Recall that $D= b^2 - 4ac$. The quadratic formula is derived from completing the square to solve for $x$ in $ax^2 + bx + c= 0$. In fact, we can derive this quickly here: we first make the $x^2$-coefficient one:
	\[
	0= ax^2 + bx + c= a \left(x^2 + \dfrac{b}{a}\,x + \dfrac{c}{a} \right)
	\]
Half the $x$-coefficient is $\frac{b}{2a}$, which squaring yields $(\frac{b^2}{4a^2}$. Adding and subtracting this term yields\dots
	\[
	\begin{aligned}
	0&= ax^2 + bx + c \\
	&= a \left(x^2 + \dfrac{b}{a}\,x + \dfrac{c}{a} \right) \\
	&= a \left(x^2 + \dfrac{b}{a}\,x + \dfrac{b^2}{4a^2} - \dfrac{b^2}{4a^2} + \dfrac{c}{a} \right) \\
	&= a \left( \left(x^2 + \dfrac{b}{a}\,x + \dfrac{b^2}{4a^2} \right) - \dfrac{b^2}{4a^2} + \dfrac{c}{a} \right) \\
	&= a \left( \left(x + \dfrac{b}{2a} \right)^2 - \dfrac{b^2}{4a^2} + \dfrac{4ac}{4a^2} \right) \\
	&= a \left( \left(x + \dfrac{b}{2a} \right)^2 + \dfrac{-b^2 + 4ac}{4a^2} \right) \\
	&= a \left(x + \dfrac{b}{2a} \right)^2 + \dfrac{-b^2 + 4ac}{4a} \\
	\end{aligned}
	\]
This is just the vertex form of $ax^2 + bx + c$. But then solving for the quadratic term, we have\dots
	\[
	\begin{gathered}
	0= a \left(x + \dfrac{b}{2a} \right)^2 + \dfrac{-b^2 + 4ac}{4a} \\
	-a \left(x + \dfrac{b}{2a} \right)^2= \dfrac{-b^2 + 4ac}{4a} \\
	a \left(x + \dfrac{b}{2a} \right)^2= -\dfrac{-b^2 + 4ac}{4a} \\
	a \left(x + \dfrac{b}{2a} \right)^2= \dfrac{b^2 - 4ac}{4a} \\
	\left(x + \dfrac{b}{2a} \right)^2= \dfrac{b^2 - 4ac}{4a^2} \\
	\sqrt{\left(x + \dfrac{b}{2a} \right)^2}= \pm \sqrt{\dfrac{b^2 - 4ac}{4a^2}} \\
	x + \dfrac{b}{2a}= \pm \dfrac{\sqrt{b^2 - 4ac}}{2a} \\
	x= -\dfrac{b}{2a} \pm \dfrac{\sqrt{b^2 - 4ac}}{2a} \\
	x= \dfrac{-b \pm \sqrt{b^2 - 4ac}}{2a} \\
	\end{gathered}
	\]



\newpage



% 02/25
\checkin{02/25} The polynomial $4 - x^2$ factors as $-(x - 2)(x + 2)$. \pspace

\sol 





























\end{document}