\documentclass[11pt,letterpaper]{article}
\usepackage[lmargin=1in,rmargin=1in,tmargin=1in,bmargin=1in]{geometry}
\usepackage{../style/quiz}
\setbool{hideans}{true} % Student: True; Instructor: False

% -------------------
% Content
% -------------------
\begin{document}
\quiz{5}

% Problem 1
\problem Consider the quadratic function $f(x)= 3 - 6x - x^2$.
	\begin{enumerate}[(a)]
	\item Find $a, b, c$ from the standard form for $f(x)$.
	\item Does the graph of $f(x)$ open upwards or downwards? Explain.
	\item Find the vertex and axis of symmetry for $f(x)$.
	\end{enumerate}

\ans{
\begin{enumerate}[(a)]
\item The standard form for a quadratic function is $ax^2 + bx + c$. Writing $f(x)$ as $f(x)= -x^2 - 6x + 3$, we see that\dots
	\[
	\boxed{a= 1, \quad b= -6, \quad c= 3}
	\]

\item Because $a= -1 < 0$, we know that the graph of $f(x)$ opens downwards. \pspace

\item We find the vertex form of $f(x)$. 

Using completing the square, we first factor out $-1$ to make the $x^2$-coefficient 1. For the new term, we find half the `middle' term ($\frac{6}{2}= 3$), square this term ($3^2= 9$) and then add/subtract this value. This yields\dots
	\[
	\begin{aligned}
	f(x)&= -x^2 - 6x + 3 \\
	&= -(x^2 + 6x - 3) \\
	&= -(x^2 + 6x + 9 - 9 - 3) \\
	&= -\big( (x^2 + 6x + 9) + (-9 - 3) \big) \\
	&= -\big( (x + 3)^2 - 12 \big) \\
	&= -(x + 3)^2 + 12
	\end{aligned}
	\]
Therefore, the vertex form is $f(x)= 12 - (x + 3)^2$. Alternatively, using the `evaluation method', we know that the vertex is located at $x= -\frac{b}{2a}= -\frac{(-6)}{2(-1)}= \frac{6}{-2}= -3$. The $y$-value at this vertex location must be $f(-3)= -(-3)^2 - 6(-3) + 3= -9 + 18 + 3= 12$. We know that $a= -1$. The vertex form is $a(x - P)^2 + Q$, where the vertex is $(P, Q)$. Therefore, we have $f(x) = -1 \big(x - (-3) \big)^2 + 12= -(x + 3)^2 + 12= 12 - (x + 3)^2$. \pspace

In either case, the vertex must be $(-3, 12)$. This also implies that the axis of symmetry is $x= -3$.
\end{enumerate}
}

\end{document}