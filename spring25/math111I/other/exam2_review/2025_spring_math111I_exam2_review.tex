\documentclass[11pt,letterpaper]{article}
\usepackage[lmargin=1in,rmargin=1in,bmargin=1in,tmargin=1in]{geometry}
\usepackage{
	amsmath,			% Math Environments
	amssymb,			% Extended Symbols
	enumerate,		% Enumerate Environments
	multicol			% Use Multiple Columns
}

% Font
\usepackage[T1]{fontenc}
\usepackage{charter}

\setlength{\parindent}{0ex}

% Tikz
\usepackage{tikz}
\usepackage{pgfplots}
\pgfplotsset{compat=newest}

% Commands
\newcounter{problem}
\newcommand{\prob}{\stepcounter{problem}%
\noindent\textbf{Problem \theproblem. }}

\newenvironment{2enumerate}{%
	\begin{enumerate}[(a)]
	\begin{multicols}{2}
	}{%
	\end{multicols}
	\end{enumerate}
}
\newcommand{\pspace}{\par\vspace{\baselineskip}}

% -------------------
% Content
% -------------------
\begin{document}

% Exam 2 Review
\begin{center} {\bfseries \LARGE Exam 2 Review} \end{center} \par\vspace{0.3cm}

% Problem 1
\prob Identify $a, b, c$ from the standard form and also find the vertex form of the following quadratic functions:
	\begin{2enumerate}
	\item $x^2 + 6x + 1$
	\item $2x^2 - 20x + 53$
	\item $x^2 - 2x + 5$
	\item $4x^2 - 4x - 5$
	\end{2enumerate} \pspace



% Problem 2
\prob Find $a, b, c$ from the standard form of the following quadratic functions. Also, find the vertex and axis of symmetry for each of them.
	\begin{enumerate}[(a)]
	\item $7 - (x + 1)^2$
	\item $x^2 - 8x + 26$
	\item $(x + 10)^2$
	\item $2(x - 7) - 15$ 
	\end{enumerate} \pspace



% Problem 3
\prob Find the vertex form of the following by using completing the square and then by using the evaluation method:
	\begin{enumerate}[(a)]
	\item $x^2 + 10x + 15$
	\item $-x^2 + 8x - 14$
	\item $3x^2 + 6x - 4$
	\end{enumerate} \pspace



% Problem 4
\prob Consider the quadratic function $f(x)= x^2 - 6x + 14$.
	\begin{enumerate}[(a)]
	\item Find $a, b, c$ for this quadratic function.
	\item Does $f(x)$ open upwards or downwards? Explain.
	\item Is this quadratic function convex or concave? Explain. 
	\item Find the vertex and axis of symmetry for $f(x)$.
	\item Find the minimum value of $f(x)$, if it exists. If it does not exist, explain why.  
	\item Find the maximum value of $f(x)$, if it exists. If it does not exist, explain why. 
	\end{enumerate} \pspace



% Problem 5
\prob Consider the quadratic function $f(x)= 12 - (x + 3)^2$.
	\begin{enumerate}[(a)]
	\item Find $a, b, c$ for this quadratic function.
	\item Does $f(x)$ open upwards or downwards? Explain.
	\item Is this quadratic function convex or concave? Explain. 
	\item Find the vertex and axis of symmetry for $f(x)$.
	\item Find the minimum value of $f(x)$, if it exists. If it does not exist, explain why.  
	\item Find the maximum value of $f(x)$, if it exists. If it does not exist, explain why. 
	\end{enumerate} \pspace



% Problem 6
\prob Consider the quadratic function $f(x)= 2x^2 - 4x + 9$.
	\begin{enumerate}[(a)]
	\item Find $a, b, c$ for this quadratic function.
	\item Does $f(x)$ open upwards or downwards? Explain.
	\item Is this quadratic function convex or concave? Explain. 
	\item Find the vertex and axis of symmetry for $f(x)$.
	\item Find the minimum value of $f(x)$, if it exists. If it does not exist, explain why.  
	\item Find the maximum value of $f(x)$, if it exists. If it does not exist, explain why. 
	\end{enumerate} \pspace



% Problem 7
\prob Showing all your work, factor each of the following as much as possible:
	\begin{2enumerate}
	\item $x^2 + 2x - 24$
	\item $x^2 - 4x + 4$
	\item $3x^2 + 24x - 27$
	\item $-2x^2 + 10x - 24$
	\item $49 - x^2$
	\item $10x^2 + 10x - 300$
	\item $x^2 + 18x + 56$
	\item $x^4 - 16$
	\item $x^2 - 2x - 120$
	\item $x^2 + x - 132$
	\end{2enumerate} \pspace



% Problem 8
\prob Solve the following quadratic equations by completing the square. Then solve the equations by using the quadratic formula. Verify your solution(s).
	\begin{enumerate}[(a)]
	\item $x^2= 2(5x - 11)$
	\item $x(2 - x)= -224$
	\item $(x - 2)(x + 2)= 2x$
	\end{enumerate} \pspace



% Problem 9
\prob Showing all your work, solve the following quadratic equations and then verify your solution(s):
	\begin{2enumerate}
	\item $x^2 - 10x + 24= 0$
	\item $15= x(x - 2)$
	\item $x^2 - x= 6$
	\item $x(6x + 1)= 15$
	\item $2x^2 + 22x= 84$
	\item $6x^2= x + 1$
	\item $88 - 4x^2= 36x$
	\item $7x - x^2= 6$
	\item $x2 + 6x - 11=0$
	\item $231x^2= 353x - 60$
	\end{2enumerate} \pspace



% Problem 10
\prob Showing all your work, solve the following equations and then verify your solution(s):
	\begin{2enumerate}
	\item $x^4 + 2x^2= 15$
	\item $\dfrac{x + 1}{x + 5}= \dfrac{x}{2x + 1}$
	\item $x - \sqrt{x} - 6= 0$
	\item $x^2 + \dfrac{4}{x^2}= 5$
	\item $\dfrac{3}{5x + 2}= x$
	\item $x^6 - 10x^3= -25$
	\end{2enumerate} \pspace



% Problem 11
\prob Showing all your work, find the domain of the following functions:
	\begin{2enumerate}
	\item $5 - (x + 6)^2$
	\item $x^5 - 4x + 12$
	\item $\dfrac{1}{7 - x}$
	\item $\sqrt{2x + 10}$
	\item $\sqrt{6 - x}$
	\item $\dfrac{\sqrt{x}}{x - 10}$
	\end{2enumerate} \pspace



% Problem 12
\prob Showing all your work, find the domain and range of the following functions:
	\begin{2enumerate}
	\item $3(x + 9)$
	\item $\dfrac{1}{x}$
	\item $1 - 4x - x^2$
	\item $\sqrt{x}$
	\item $(x + 5)^2 - 11$
	\item $6x + 17$
	\end{2enumerate} \pspace



% Problem 13
\prob Without explicitly solving the following equations, determine whether there is a solution. Be sure to fully justify your reasoning. 
	\begin{enumerate}[(a)]
	\item $2x^2 + 5x= 3$
	\item $x^2 + 7= x$
	\item $(3 - x)(3 + x)= 8x$
	\end{enumerate} \pspace



% Problem 14
\prob Without explicitly solving the following equations, determine whether the given quadratic can be factored. Be sure to fully justify your reasoning. 
	\begin{enumerate}[(a)]
	\item $x^2 + 4x - 45$
	\item $x^2 + 3x - 1$
	\item $6x^2 - 8x + 2$
	\end{enumerate} \pspace



% Problem 15
\prob Use the quadratic formula to factor the following:
	\begin{2enumerate}
	\item $x^2 - 4x - 32$
	\item $24x^2 + 26 - 5$
	\item $24x^2 - 26x - 15$
	\item $72x^2 + 71x - 120$
	\end{2enumerate} \pspace



% Problem 16
\prob Values for several functions are given in the table below. 
        \begin{table}[!ht]
        \centering
        \begin{tabular}{| c || c | c | c | c | c | c | c |} \hline
	$x$ & $-3$ & $-2$ & $-1$ & $\phantom{-}0$ & $\phantom{-}1$ & $\phantom{-}2$ & $\phantom{-}3$ \\ \hline \hline
	$f(x)$ & $\phantom{-}4$ & $8$ & $-1$ & $\phantom{-}5$ & $-3$ & $\phantom{-}0$ & $-2$ \\ \hline
	$g(x)$ & $\phantom{-}1$ & $6$ & $\phantom{-}0$ & $-6$ & $-7$ & $-3$ & $\phantom{-}1$ \\ \hline
	$h(x)$ & $-4$ & $0$ & $\phantom{-}3$ & $\phantom{-}5$ & $10$ & $\phantom{-}3$ & $\phantom{-}9$ \\ \hline
        \end{tabular}
        \end{table}

Given the data above, compute the following: 
        \begin{2enumerate}
        \item $(h + g)(-2)=$ \vfill
        \item $(f - g)(0)=$ \vfill
        \item $(5h)(1)=$ \vfill
        \item $\left(\dfrac{h}{f}\right)(1)=$ \vfill
        \item $g(-3)\, h(3)=$ \vfill
        \item $g \big(-1 - f(3) \big)=$ \vfill
        \item $(h \circ g)(2)=$ \vfill
	\item $(g \circ h)(2)=$ \vfill
        \item $(f \circ g)(-1)=$ \vfill
	\item $(h \circ g \circ f)(1)=$ \vfill
        \end{2enumerate} \pspace



% Problem 17
\prob Write each of the following functions as a composition of functions $f \big( g(x) \big)$.
	\begin{2enumerate}
	\item $(x - 5)^2 + 8$
	\item $\dfrac{1}{x + 8}$
	\item $(1 - x)^{11}$
	\item $\dfrac{1}{(3x + 5)^4}$
	\item $\sqrt{x^2 + 10}$
	\item $\dfrac{1}{2(x - 5)^3}$
	\end{2enumerate} \pspace



% Problem 18
\prob Suppose $f(x)$ and $g(x)$ are the functions given below. 
	\[
	\begin{aligned}
	f(x)&= 2x - 3 \\[0.3cm]
	g(x)&= x^2 + 2x - 1
	\end{aligned}
	\]

Compute the following: \pspace
        \begin{2enumerate}
        \item $f(5)=$ \vfill
        \item $g(-2)=$ \vfill
        \item $f(0) - 3g(2)=$ \vfill
        \item $(f - g)(x)=$ \vfill
        \item $(fg)(x)=$ \vfill
        \item $\left( \dfrac{f}{g} \right)(x)=$ \vfill
        \item $(f \circ g)(0)=$ \vfill
        \item $(g \circ f)(0)=$ \vfill
        \item $(f \circ g)(x)=$ \vfill
        \item $(g \circ f)(x)=$ \vfill
        \end{2enumerate} \pspace



% Problem 19
\prob Let $f(x)$ be the function given by $f(x)= 3x - 7$. 
	\begin{enumerate}[(a)]
	\item Find a value in the range of $f$. Be sure to justify why the value is in the range. 
	\item Compute $f(4)$. Is $(4, 1)$ on the graph of $f$? Explain. 
	\item Is there an $x$ such that $f(x)= 11$? Explain. 
	\item Is $1 \in f^{-1}(3)$? Explain. 
	\item Assuming $f^{-1}$ exists, what is $f(f^{-1}(\pi))$ and $f^{-1}(f(\sqrt{2}))$?
	\end{enumerate} \pspace



% Problem 20
\prob Suppose $f(x)$ and $g(x)$ are the functions given below. 
        \begin{table}[!ht]
        \centering
        \begin{tabular}{| c || c | c | c | c | c | c | c |} \hline
	$x$ & $-3$ & $-2$ & $-1$ & $\phantom{-}0$ & $\phantom{-}1$ & $\phantom{-}2$ & $\phantom{-}3$ \\ \hline
	$f(x)$ & $5$ & $2$ & $0$ & $-1$ & $-2$ & $-4$ & $-5$ \\ \hline
	$g(x)$ & $1$ & $1$ & $5$ & $2$ & $-3$ & $-3$ & $4$ \\ \hline
	$h(x)$ & $-6$ & $7$ & $1$ & $-2$ & $0$ & $1$ & $-1$ \\ \hline
        \end{tabular}
        \end{table}

Compute the following: \pspace
        \begin{2enumerate}
        \item $(f + g)(3)=$ \vfill
        \item $(f - g)(-1)=$ \vfill
        \item $(5h)(1)=$ \vfill
        \item $\left(\dfrac{h}{g}\right)(-3)=$ \vfill
        \item $f(2)\, h(-2)=$ \vfill
        \item $h(-1 - f(0))=$ \vfill
        \item $(g \circ f)(-2)=$ \vfill
	\item $(h \circ g)(1)=$ \vfill
        \item $(g \circ h)(1)=$ \vfill
	\item $(g \circ f \circ h)(-1)=$ \vfill
        \end{2enumerate} \pspace



% Problem 21
\prob Suppose $f(x)$ and $g(x)$ are the functions given below. 
	\[
	\begin{aligned}
	f(x)&= 3x - 10 \\[0.3cm]
	g(x)&= 2x^2 - x + 5
	\end{aligned}
	\]

Compute the following: \pspace
\begin{2enumerate}
\item $f(3)=$ \vfill
\item $g(-2)=$ \vfill
\item $5f(6) - g(1)=$ \vfill
\item $f(x) - g(x)=$ \vfill
\item $f(x) \, g(x)=$ \vfill
\item $\left( \dfrac{f}{g} \right)(x)=$ \vfill
\item $(f \circ g)(0)=$ \vfill
\item $(g \circ f)(3)=$ \vfill
\item $(f \circ g)(x)=$ \vfill
\item $(g \circ f)(x)=$ \vfill
\end{2enumerate} \pspace



% Problem 22
\prob Let $f(x)= x^2$, $g(x)= \dfrac{1}{x}$, $h(x)= \sqrt{x}$, and $k(x)= 3x + 1$. For each of the given functions, determine the function described:
	\begin{enumerate}[(a)]
	\item the function shifted two up and three to the right.
	\item the function shifted 5 to the left and 4 down.
	\item the function reflected through the $x$ and $y$-axis.
	\item the function compressed in the $x$-directed by a factor of 3.
	\item the function scaled vertically by a factor of 2 and then shifted 5 downward.
	\item the function shifted two upwards, reflected across the $y$-axis, and then shifted two right.
	\item the function shifted 6 down, 5 to the left, and then reflected across the $x$-axis.
	\end{enumerate} \pspace



% Problem 23
\prob A function $f(x)$ is plotted in black. Determine the other functions plotted in terms of $f(x)$.
	\[
	\fbox{
	\begin{tikzpicture}[scale=0.7,every node/.style={scale=0.5}]
	\begin{axis}[
	grid=both,
	axis lines=middle,
	ticklabel style={fill=blue!5!white},
	xmin= -10.5, xmax=10.5,
	ymin= -10.5, ymax=10.5,
	xtick={-10,-8,...,10},
	ytick={-10,-8,...,10},
	minor tick = {-10,-9,...,10},
	xlabel=\(x\),ylabel=\(y\),
	]
	\addplot[thick, domain= -10.5:10.5, samples=100] ({x},{x^2});
	
	\addplot[thick, domain= -10.5:10.5, samples=100,red] ({x},{(x + 6)^2 - 3});
	\addplot[thick, domain= -10.5:10.5, samples=100,blue] ({x},{-x^2 - 1});
	\addplot[thick, domain= -10.5:10.5, samples=100,green] ({x},{5 - (x - 8)^2});
	\addplot[thick, domain= -10.5:10.5, samples=100,purple] ({x},{(x - 4)^2 + 3});
	\end{axis}
	\end{tikzpicture}
	}
	\] \pspace



% Problem 24
\prob A function $f(x)$ is plotted in black. Determine the other functions plotted in terms of $f(x)$.
	\[
	\fbox{
	\begin{tikzpicture}[scale=0.7,every node/.style={scale=0.5}]
	\begin{axis}[
	grid=both,
	axis lines=middle,
	ticklabel style={fill=blue!5!white},
	xmin= -10.5, xmax=10.5,
	ymin= -10.5, ymax=10.5,
	xtick={-10,-8,...,10},
	ytick={-10,-8,...,10},
	minor tick = {-10,-9,...,10},
	xlabel=\(x\),ylabel=\(y\),
	]
	\addplot[thick, domain= 0:10.5, samples=100] ({x},{sqrt(x)});
	
	\addplot[thick, domain= -3:10.5, samples=100,red] ({x},{-sqrt(x + 3) - 2});
	\addplot[thick, domain= -10.5:4, samples=100,blue] ({x},{sqrt(4 - x)});
	\addplot[thick, domain= -10.5:0, samples=100,orange] ({x},{sqrt(-x) + 4});
	\end{axis}
	\end{tikzpicture}
	}
	\] \pspace



% Problem 25
\prob If $f(x)$ is a function, describe the graph of the given function in terms of the graph of $f(x)$.
	\begin{2enumerate}
	\item $f(x + 3) - 4$
	\item $5 - f(x - 1)$
	\item $-f(-x)$
	\item $f(-x + 3)$
	\item $f(3x)$
	\item $-6f(x + 5)$
	\item $f(-x) + 7$
	\item $2f(1 - x) + 1$
	\end{2enumerate} \pspace



% Problem 26
\prob Determine if the relation below is a function or not. If it is a function, explain why. If it is not a function, explain why. Determine also whether the relation has an inverse function. If it has an inverse function, explain why and sketch it. If it does not have an inverse function, explain why not. 
	\[
	\fbox{
	\begin{tikzpicture}[scale=1,every node/.style={scale=0.5}]
	\begin{axis}[
	grid=both,
	axis lines=middle,
	ticklabel style={fill=blue!5!white},
	xmin= -7, xmax=7,
	ymin= -6.5, ymax=6.5,
	xtick={-6,-4,-2,0,2,4,6},
	ytick={-6,-4,-2,0,2,4,6},
	minor tick = {-5,-3,...,5},
	xlabel=\(x\),ylabel=\(y\),
	]
	\node at (5.5,2.3) {$f(x)$};
	\addplot[thick, domain= -7:7, samples=100] ({x},{5 - (x + 1)^2});
	\end{axis}
	\end{tikzpicture}
	}
	\] \pspace
	
	

% Problem 27
\prob Determine if the relation below is a function or not. If it is a function, explain why. If it is not a function, explain why. Determine also whether the relation has an inverse function. If it has an inverse function, explain why and sketch it. If it does not have an inverse function, explain why not. 
	\[
	\fbox{
	\begin{tikzpicture}[scale=1,every node/.style={scale=0.5}]
	\begin{axis}[
	grid=both,
	axis lines=middle,
	ticklabel style={fill=blue!5!white},
	xmin= -7, xmax=7,
	ymin= -6.5, ymax=6.5,
	xtick={-6,-4,-2,0,2,4,6},
	ytick={-6,-4,-2,0,2,4,6},
	minor tick = {-5,-3,...,5},
	xlabel=\(x\),ylabel=\(y\),
	]
	\node at (5.5,2.3) {$f(x)$};
	\addplot[thick, domain= -7:7, samples=100] ({x},{1/10*(x + 5)*(x + 1)*(x - 3)});
	\end{axis}
	\end{tikzpicture}
	}
	\] \pspace



\newpage



% Problem 28
\prob Determine if the relation below is a function or not. If it is a function, explain why. If it is not a function, explain why. Determine also whether the relation has an inverse function. If it has an inverse function, explain why and sketch it. If it does not have an inverse function, explain why not. 
	\[
	\fbox{
	\begin{tikzpicture}[scale=1,every node/.style={scale=0.5}]
	\begin{axis}[
	grid=both,
	axis lines=middle,
	ticklabel style={fill=blue!5!white},
	xmin= -10.5, xmax=10.5,
	ymin= -10.5, ymax=10.5,
	xtick={-10,-8,-6,-4,-2,0,2,4,6,8,10},
	ytick={-10,-8,-6,-4,-2,0,2,4,6,8,10},
	minor tick = {-10,-9,...,10},
	xlabel=\(x\),ylabel=\(y\),
	]
	\addplot[line width= 0.02cm,samples=100,domain= -10.5:-7] ({x},{-(x + 7)^2 - 3}); 
	\addplot[line width= 0.02cm,samples=100,domain= -7:0] ({x},{3/7*x}); 
	\addplot[line width= 0.02cm,samples=100,domain= 0:10.5] ({x},{x^2/10}); 
	\end{axis}
	\end{tikzpicture}
	}
	\] \pspace



% Problem 29
\prob Find the inverse to the following functions. Also, verify that your inverse is correct:
	\begin{2enumerate}
	\item $2x - 1$
	\item $\frac{1}{3} \,x + 5$
	\item $\dfrac{6x - 9}{-2}$
	\item $8(3 - x)$
	\end{2enumerate} \pspace



% Problem 30
\prob Determine whether the following statements are true or false. Be sure to justify your answer:
	\begin{enumerate}[(a)]
	\item A function has an inverse if it passes the HLT.
	\item A function can have more than one inverse.
	\item All functions have an inverse.
	\item The function $x^2 + 8$ has an inverse.
	\item If $f$ has an inverse, then $f(f^{-1}(\pi))= \pi$.
	\item All lines have an inverse.
	\end{enumerate}

\end{document}