\documentclass[11pt,letterpaper]{article}
\usepackage[lmargin=1in,rmargin=1in,bmargin=1in,tmargin=1in]{geometry}
\usepackage{checkins}

\DeclareMathOperator{\arccot}{arccot}
\newcommand{\boxseven}[4]{%
	\draw[thick] (0,0) -- (4,0) -- (4,4) -- (0,4) -- (0,0);
	\draw[thick] (0,2) -- (4,2);
	\draw[thick] (2,0) -- (2,4);
	% '7'
	\draw (1.7,2.2) -- (2.3,2.2) -- (1.7,1.6);
	% Entries
	\node at (1,3) {$#1$};	% u
	\node at (3,1) {$#2$};	% dv
	\node at (1,1) {$#3$};	% du
	\node at (3,3) {$#4$};	% 
}

\usetikzlibrary{calc}
\usepackage{booktabs}
\tikzset{Arrow Style/.style={text=black, font=\boldmath}}
\newcommand{\tikzmark}[1]{%
    \tikz[overlay, remember picture, baseline] \node (#1) {};%
}
\newcommand*{\XShift}{0.5em}
\newcommand*{\YShift}{0.5ex}
\NewDocumentCommand{\DrawArrow}{s O{} m m m}{%
    \begin{tikzpicture}[overlay,remember picture]
        \draw[->, thick, Arrow Style, #2] 
                ($(#3.west)+(\XShift,\YShift)$) -- 
                ($(#4.east)+(-\XShift,\YShift)$)
        node [midway,above] {#5};
    \end{tikzpicture}%
}
\usepackage{polynom}

% -------------------
% Content
% -------------------
\begin{document}
\thispagestyle{title}

% 01/16
\checkin{01/16} Given $\ds\int_0^\pi e^{\sin x} \cos x \;dx$, the $u$-substitution $u= \sin x$ transforms this integral into $\ds\int_0^\pi e^u \;du$. \pspace

\sol The statement is \textit{false}. If $u= \sin x$, then $du= \cos x \,dx$. So indeed, this $u$-substitution would transform the integral $\ds\int e^{\sin x} \cos x \,dx$ into the integral $\ds\int e^u \,du$. However with definite integrals, one needs to remember to transform the limits as well. If $x= 0$, then $u= \sin(0)= 0$. If $x= \pi$, then $u= \sin(\pi)= 0$. Therefore, the correct substitution is $\ds\int_0^\pi e^{\sin x} \cos x \,dx= \int_0^0 e^u \,du= 0$. \pvspace{1.3cm}



% 01/21
\checkin{01/21} To integrate $\ds\int \arccot \theta \,d\theta$, one can use integration-by-parts by choosing $u= \arccot \theta$ and $dv= 1$. \pspace

\sol The statement is \textit{true}. Using LIATE, it is likely that the choice of $u= \arccot \theta$ will work. With `nothing left' in the integrand, this means that $dv= 1$. We fill in our box as follows:
	\[
	\begin{tikzpicture}
	% Left Boxes
	\draw[thick] (0,0) -- (4,0) -- (4,4) -- (0,4) -- (0,0);
	\draw[thick] (0,2) -- (4,2);
	\draw[thick] (2,0) -- (2,4);
	% Entries
	\node at (1,3) {$\arccot \theta$};	% u
	\node at (3,1) {$1$};				% dv
	
	\tikzset{shift={(3,0)}}
	
	\node at (2,2) {\scalebox{1.5}{$\Longrightarrow$}};
	
	\tikzset{shift={(3,0)}}
	
	\boxseven{\arccot \theta}{1}{\dfrac{-1}{1 + \theta^2}}{\theta}
	\end{tikzpicture}
	\]
Then using the `Rule of 7', we find that\dots
	\[
	\int \arccot \theta \,d\theta= \theta \arccot \theta - \int \dfrac{-\theta}{1 + \theta^2} \,d\theta= \theta \arccot \theta + \int \dfrac{\theta}{1 + \theta^2} \,d\theta
	\]
Using the $u$-substitution $u= 1 + \theta^2$, we see that $\ds\int \frac{\theta}{1 + \theta^2} \,d\theta= \frac{1}{2} \ln|1 + \theta^2| + C$. Therefore, we have\dots
	\[
	\int \arccot \theta \,d\theta= \theta \arccot \theta + \dfrac{1}{2} \ln|1 + \theta^2| + C
	\] \pvspace{1.3cm}



% 01/23
\checkin{01/23} The integral $\ds\int e^{x/2} \sin(3x) \;dx$ is a `looping' integral. \pspace

\sol The statement is \textit{true}. Recall that integrals whose integrand is the product of an exponential function with $\sin x$ or $\cos x$ `loop.' We can see this directly: choose $u= \sin(3x)$. Using the `box method', we have\dots
	\[
	\begin{tikzpicture}
	\boxseven{\sin(3x)}{e^{x/2}}{3 \cos(3x)}{2e^{x/2}}
	\end{tikzpicture}
	\]
Therefore, we have\dots
	\[
	\int e^{x/2} \sin(3x) \;dx= 2 e^{x/2} \sin(3x) - \int 6e^{x/2} \cos(3x) \;dx
	\]
But this integral on the right also requires integration-by-parts: we choose $u= \cos(3x)$ and then\dots
	\[
	\begin{tikzpicture}
	\boxseven{\cos(3x)}{6 e^{x/2}}{-3 \sin(3x)}{12 e^{x/2}}
	\end{tikzpicture}
	\]
So then we have\dots
	\[
	\begin{aligned}
	\int e^{x/2} \sin(3x) \;dx&= 2 e^{x/2} \sin(3x) - \int 6e^{x/2} \cos(3x) \;dx \\
	&= 2 e^{x/2} \sin(3x) - \left(12 e^{x/2} \cos(3x) - \int -36 e^{x/2} \sin(3x) \;dx \right) \\
	&= 2 e^{x/2} \sin(3x) - 12 e^{x/2} \cos(3x) + \int -36 e^{x/2} \sin(3x) \;dx \\
	&= 2 e^{x/2} \sin(3x) - 12 e^{x/2} \cos(3x) - 36 \int e^{x/2} \sin(3x) \;dx
	\end{aligned}
	\]
Observe that we have `looped'---obtaining a multiple of the original integral on the right. Adding $\ds 36 \int e^{x/2} \sin(3x) \;dx$ to both sides, we have\dots
	\[
	37 \int e^{x/2} \sin(3x) \;dx= 2 e^{x/2} \sin(3x) - 12 e^{x/2} \cos(3x)
	\]
Therefore, we have\dots
	\[
	\int e^{x/2} \sin(3x) \;dx= \dfrac{2 e^{x/2} \sin(3x) - 12 e^{x/2} \cos(3x)}{37} + C
	\]
We can shortcut this work by adjusting tabular integration: 
	\[
	\renewcommand{\arraystretch}{1.5}
	\begin{array}{c @{\hspace*{1.3cm}} c} \toprule
	u & dv \\ \cmidrule(lr){1-2}
	\sin(3x) \tikzmark{Left 1} & \tikzmark{Right 1} e^{x/2} \\[0.3cm]
	3 \cos(3x) \tikzmark{Left 2} & \tikzmark{Right 2} 2 e^{x/2} \\[0.3cm]
	-9 \sin(3x) \tikzmark{Left 3}  & \tikzmark{Right 3} 4 e^{x/2} 
	
	\DrawArrow{Left 1}{Right 2}{+}
	\DrawArrow{Left 2}{Right 3}{--}
	\DrawArrow{Left 3}{Right 3}{+}
	\end{array}
	\]
Therefore, we have\dots
	\[
	\begin{aligned}
	\int e^{x/2} \sin(3x) \;dx&= 2 e^{x/2} - 12 e^{x/2} \cos(3x) - 36 \int e^{x/2} \sin(3x) \; dx \\
	37 \int e^{x/2} \sin(3x) \;dx&= 2 e^{x/2} - 12 e^{x/2} \cos(3x) \\
	\int e^{x/2} \sin(3x) \;dx&= \dfrac{2 e^{x/2} \sin(3x) - 12 e^{x/2} \cos(3x)}{37} + C
	\end{aligned}
	\] \pvspace{1.3cm}



% 01/28
\checkin{01/28} To integrate $\ds\int \cos^8 \theta \sin^5 \theta \;d\theta$, one should choose $u= \cos \theta$. \pspace

\sol The statement is \textit{true}. Observe that if we choose $u= \cos \theta$, then $\cos^8 \theta$ becomes $u^8$. We know $du$ will then produce a $\sin \theta$---specifically $-\sin \theta$. This `uses' one of the $\sin \theta$'s in the integrand. This leaves $\sin^4 \theta$ remaining in the integrand. But we can replace even powers of $\sin \theta$ in terms of $\cos \theta$. So we can carry out this substitution. Observe that if $u= \cos \theta$, then $du= -\sin \theta \;d\theta$. But then using the fact that $\sin^2 \theta= 1 - \cos^2 \theta$ (so that $\sin^4 \theta= (\sin^2 \theta)^2= (1 - \cos^2 \theta)^2$), we have\dots
	\[
	\begin{aligned}
	\int \cos^8 \theta \sin^5 \theta \;d\theta&= \int \cos^8 \theta \sin^4 \theta \cdot \sin \theta \; d\theta \\
	&= -\int \cos^8 \theta \sin^4 \theta \cdot -\sin \theta \; d\theta \\
	&= -\int \cos^8 \theta (1 - \cos^2 \theta)^2 \cdot -\sin \theta \;d\theta \\
	&= -\int u^8 (1 - u^2)^2 \;du \\
	&= -\int u^8 (1 - 2u^2 + u^4) \;du \\
	&= \int -u^8 + 2u^{10} - u^{12} \;du \\
	&= -\dfrac{u^9}{9} + \dfrac{2u^{11}}{11} - \dfrac{u^{13}}{13} + C \\
	&= -\dfrac{\cos^9 \theta}{9} + \dfrac{2\cos^{11} \theta}{11} - \dfrac{\cos^{13} \theta}{13} + C
	\end{aligned}
	\]
Alternatively, observe that if we had chosen $u= \sin \theta$, then $\sin^5 \theta$ becomes $u^5$. We know that $du$ produces a $\cos \theta$. This `uses' one of the $\cos \theta$'s in the integrand. This leaves $\cos^7 \theta$ remaining in the integrand. However, we can only replace even powers of $\cos \theta$ in terms of $\sin \theta$ using $\cos^2 \theta= 1 - \sin^2 \theta$. So without using other identities or using other techniques, we cannot `simply' choose $u= \sin \theta$ for this integral. \pvspace{1.3cm}



% 01/30
\checkin{01/30} To compute $\ds\int \dfrac{x^3}{\sqrt{x^2 - 4}} \;dx$, one can make the substitution $x= 2 \sec \theta$. \pspace

\sol The statement is \textit{true}. Observe we have the term $x^2 - 4$, which resembles a term from the Pythagorean Theorem. For a right triangle, we know that $a^2 + b^2= c^2$. This implies that $a^2= c^2 - b^2$, which is $x^2 - 4$ if $c^2= x^2$ and $b^2= 4$, i.e. $c= x$ and $b= 2$. We construct a triangle with $c= x$ and $b= 2$. This only gives two possibilities:
	\[
	\begin{tikzpicture}
	\draw[line width=0.03cm] (0,0) -- (1.5,0) -- (1.5,2) -- (0,0);
	\draw[line width=0.02cm] (1.3,0) -- (1.3,0.2) -- (1.5,0.2);
	\node at (0.45,0.25) {$\theta$};
	\node at (0.55,1.1) {$x$};
	\node at (0.75,-0.3) {$2$};
	
	\tikzset{shift={(4,0)}}

	\draw[line width=0.03cm] (0,0) -- (1.5,0) -- (1.5,2) -- (0,0);
	\draw[line width=0.02cm] (1.3,0) -- (1.3,0.2) -- (1.5,0.2);
	\node at (0.45,0.25) {$\theta$};
	\node at (0.55,1.1) {$x$};
	\node at (1.8,0.9) {$2$};
	\end{tikzpicture}
	\]
In the first, we would have $\cos \theta= \frac{2}{x}$, which implies that $x= \frac{2}{\cos \theta}= 2 \sec \theta$. This is the given substitution. In the second, we would have $\sin \theta= \frac{2}{x}$, which implies that $x= \frac{2}{\sin \theta}= 2 \csc \theta$. 

Choosing the former substitution, we would have $x= 2 \sec \theta$, so that $dx= 2 \sec \theta \tan \theta$. Calling the vertical side $s$ and using the Pythagorean Theorem, we see that $2^2 + s^2= x^2$, i.e. $4 + s^2= x^2$. But then $s= \sqrt{x^2 - 4}$. This gives us the following triangle: 
	\[
	\begin{tikzpicture}
	\draw[line width=0.03cm] (0,0) -- (1.5,0) -- (1.5,2) -- (0,0);
	\draw[line width=0.02cm] (1.3,0) -- (1.3,0.2) -- (1.5,0.2);
	\node at (0.45,0.25) {$\theta$};
	\node at (0.55,1.1) {$x$};
	\node at (0.75,-0.3) {$2$};
	\node at (2.4,0.9) {$\sqrt{x^2 - 4}$};
	\end{tikzpicture}
	\]
We need to find $x^3$ and $\sqrt{x^2 - 4}$. Because $x= 2 \sec \theta$, we know that $x^3= (2 \sec \theta)^3= 2^3 \sec^3 \theta$. To find $\sqrt{x^2 - 4}$, observe that $\tan \theta= \frac{\sqrt{x^2 - 4}}{2}$, which implies that $\sqrt{x^2 - 4}= 2 \tan \theta$. Therefore, we have\dots
	\[
	\int \dfrac{x^3}{\sqrt{x^2 - 4}} \;dx= \int \dfrac{2^3 \sec^3 \theta}{2 \tan \theta} \cdot 2 \sec \theta \tan \theta \;d\theta= \int 8 \sec^4 \theta \;d\theta
	\]
Now using the fact that $\tan^2 \theta + 1= \sec^2 \theta$, we know that 
	\[
	\sec^4 \theta= (\sec^2 \theta)^2= (\tan^2 \theta + 1)^2= \tan^4 \theta + 2 \tan^2 \theta + 1
	\]
But then\dots
	\[
	\int 8 \sec^4 \theta \;d\theta= \int 8 \sec^2 \theta \sec^2 \theta \;d\theta= \int 8 \sec^2 \theta (\tan^2 \theta + 1) \;d\theta= \int 8(\tan^2 \theta + 1) \cdot \sec^2 \theta \;d\theta
	\]
Now letting $u= \tan \theta$, we have $du= \sec^2 \theta \;d\theta$. But then\dots
	\[
	\int 8(\tan^2 \theta + 1) \cdot \sec^2 \theta \;d\theta= \int 8(u^2 + 1) \;du= 8 \int (u^2 + 1) \;du= 8 \left( \dfrac{u^3}{3} + u \right) + C
	\]
Replacing $u$, we have\dots
	\[
	8 \left( \dfrac{u^3}{3} + u \right) + C= 8 \left( \dfrac{\tan^3 \theta}{3} + \tan \theta \right) + C
	\]
But from the triangle, we know that $\tan \theta= \frac{\sqrt{x^2 - 4}}{2}$. Therefore, we have\dots
	\[
	8 \left( \dfrac{\tan^3 \theta}{3} + \tan \theta \right) + C=. 8 \left( \dfrac{1}{3} \cdot \left( \dfrac{\sqrt{x^2 - 4}}{2} \right)^3 + \dfrac{\sqrt{x^2 - 4}}{2} \right) + C
	\]
If we simplify this, we have\dots
	\[
	\begin{gathered}
	8 \left( \dfrac{1}{3} \cdot \left( \dfrac{\sqrt{x^2 - 4}}{2} \right)^3 + \dfrac{\sqrt{x^2 - 4}}{2} \right) + C \\
	8 \cdot \dfrac{\sqrt{x^2 - 4}}{2} \left( \dfrac{1}{3} \cdot \left( \dfrac{\sqrt{x^2 - 4}}{2} \right)^2 + 1 \right) + C \\
	4 \sqrt{x^2 - 4} \left( \dfrac{1}{3} \cdot \dfrac{x^2 - 4}{4} + 1 \right) + C \\
	4 \sqrt{x^2 - 4} \left( \dfrac{x^2 - 4}{12} + 1 \right) + C \\
	4 \sqrt{x^2 - 4} \left( \dfrac{x^2 - 4}{12} + \dfrac{12}{12} \right) + C \\
	4 \sqrt{x^2 - 4} \left( \dfrac{x^2 + 8}{12}\right) + C \\
	\dfrac{1}{3}\, \sqrt{x^2 - 4}\; (x^2 + 8) + C
	\end{gathered}
	\]
Therefore, we have\dots
	\[
	\int \dfrac{x^3}{\sqrt{x^2 - 4}} \;dx= \dfrac{1}{3}\, \sqrt{x^2 - 4}\; (x^2 + 8) + C
	\] \pvspace{1.3cm}



% 02/04
\checkin{02/04} The rational function $\dfrac{x^6 - 4x}{(x - 1)(x + 2)(x^2 + 4)}$ can be decomposed as $\dfrac{A}{x - 1} + \dfrac{B}{x + 2} + \dfrac{Cx + D}{x^2 + 4}$. \pspace

\sol The statement is \textit{false}. We know that given a rational function has a partial fraction decomposition given in the `traditional' way so long as the degree of the numerator is \textit{less} than the degree of the denominator. Observe that the degree of the numerator in the original function is 6 while the degree of the denominator is $1 + 1 + 2= 4$. So, this cannot be broken down in the `traditional' way. Alternatively, find the common denominator of $(x - 1)(x + 2)(x^2 + 4)$ for the given decomposition, observe that the numerator will have at most degree 4, which could not possibly yield the numerator of $x^6 - 4x$ with degree 6. One would first need to long divide $x^6 - 4x$ by $(x - 1)(x + 2)(x^2 + 4)$ to find the quotient and remainder before trying to give a partial fraction decomposition. Observe\dots
	\[
	(x - 1)(x + 2)(x^2 + 4)= (x^2 + x - 2)(x^2 + 4)= x^4 + x^3 + 2x^2 + 4x - 8
	\]
But then\dots
	\[
	\polylongdiv{x^6 - 4x}{x^4 + x^3 + 2x^2 + 4x - 8}
	\]
Therefore, we have\dots
	\[
	\dfrac{x^6 - 4x}{(x - 1)(x + 2)(x^2 + 4)}= x^2 - x - 1 + \dfrac{-x^3 + 14x^2 - 8x - 8}{x^4 + x^3 + 2x^2 + 4x - 8}= x^2 - x - 1 + \dfrac{-x^3 + 14x^2 - 8x - 8}{(x - 1)(x + 2)(x^2 + 4)}
	\]
We can then find the `traditional' partial fraction decomposition of the resulting rational function above because the degree of the numerator, which is 3, is \textit{less} than the degree of the denominator, which is 4. We would have\dots
	\[
	\dfrac{-x^3 + 14x^2 - 8x - 8}{(x - 1)(x + 2)(x^2 + 4)}= \dfrac{A}{x - 1} + \dfrac{B}{x + 2} + \dfrac{Cx + D}{x^2 + 4}
	\]
Using Heaviside's Method/Cover-Up Method, we can find $A$ and $B$ `instantly':
	\[
	\begin{aligned}
	A&= \dfrac{-1^3 + 14(1^2) - 8(1) - 8}{(1 + 2)(1^2 + 4)}= -\dfrac{3}{15}= -\dfrac{1}{5} \\[0.2cm]
	B&= \dfrac{-(-2^3) + 14(-2)^2 - 8(-2) - 8}{(-2 - 1) \big( (-2)^2 + 4 \big)}= \dfrac{72}{-24}= -3
	\end{aligned}
	\]
To find $C, D$, we can get a common denominator:
	\[
	\dfrac{A}{x - 1} + \dfrac{B}{x + 2} + \dfrac{Cx + D}{x^2 + 4}= \dfrac{A(x + 2)(x^2 + 4) + B(x - 1)(x^2 + 4) + (Cx + D)(x - 1)(x + 2)}{(x - 1)(x + 2)(x^2 + 4)}
	\]
Using the fact that the rational functions now have equal denominators, we can equate their numerators:
	\[
	\begin{aligned}
	-x^3 + 14x^2 - 8x - 8&= A(x + 2)(x^2 + 4) + B(x - 1)(x^2 + 4) + (Cx + D)(x - 1)(x + 2) \\[0.2cm]
	-x^3 + 14x^2 - 8x - 8&=-\frac{1}{5}\, (x + 2)(x^2 + 4) - 3(x - 1)(x^2 + 4) + (Cx + D)(x - 1)(x + 2) \\
	\end{aligned}
	\]
We multiply the last equation by 5 to clear fractions:
	\[
	5(-x^3 + 14x^2 - 8x - 8)= -(x + 2)(x^2 + 4) - 15(x - 1)(x^2 + 4) + 5(Cx + D)(x - 1)(x + 2)
	\]
Rather than expand this and relate coefficients to obtain a system of equations, we will simply substitute two $x$-values (ones not used in Heaviside's Method): using $x= 0$, we have\dots
	\[
	\begin{aligned}
	5(-x^3 + 14x^2 - 8x - 8)&= -(x + 2)(x^2 + 4) - 15(x - 1)(x^2 + 4) + 5(Cx + D)(x - 1)(x + 2) \\
	-40&= 52 - 10D \\
	-92&= -10 D \\
	\dfrac{46}{5}&= D
	\end{aligned}
	\]
and using $x= -1$ and the value for $D$ above, we have\dots
	\[
	\begin{aligned}
	5(-x^3 + 14x^2 - 8x - 8)&= -(x + 2)(x^2 + 4) + 15(x - 1)(x^2 + 4) + 5(Cx + D)(x - 1)(x + 2) \\
	75&= 145 - 10 (-C + D) \\
	-70&= 10C - 10D \\
	-70&= 10C - 10 \cdot \dfrac{46}{5} \\
	-70&= 10C - 92 \\
	22&= 10C \\
	\dfrac{11}{5}&= C
	\end{aligned}
	\]
Therefore, we have\dots
	\[
	\dfrac{x^6 - 4x}{(x - 1)(x + 2)(x^2 + 4)}= x^2 - x - 1 + \dfrac{-1/5}{x - 1} + \dfrac{-3}{x + 2} + \dfrac{\frac{11}{5}\,x + \frac{46}{5}}{x^2 + 4}= x^2 - x - 1 - \dfrac{1}{5(x - 1)} - \dfrac{3}{x + 2} + \dfrac{11x + 46}{5(x^2 + 4)}
	\] \pvspace{1.3cm}



% 02/06
\checkin{02/06} 
	\[
	\ds\int \dfrac{5}{x + 1} - \dfrac{1}{(x + 1)^2} - \dfrac{x + 1}{x^2 + 1} \;dx= 5 \ln(x + 1) - \dfrac{1}{x + 1} - \frac{1}{2} \ln|x^2 + 1| + \arctan x + K
	\] \pspace

\sol The statement is \textit{false}. First, we know that\dots
	\[
	\int \dfrac{5}{x + 1} \;dx= 5 \ln|x + 1| + K
	\]
The integral of the second term is also incorrect:
	\[
	\int -\dfrac{1}{(x + 1)^2} \;dx= \int -(x + 1)^{-2} \;dx= \dfrac{-(x + 1)^{-1}}{-1} + K= \dfrac{1}{x + 1} + K
	\]
Moreover, the third term is also incorrect as the negative has been improperly distributed:
	\[
	- \int \dfrac{x + 1}{x^2 + 1} \;dx= -\int \dfrac{x}{x^2 + 1} + \dfrac{1}{x^2 + 1} \;dx= - \left( \frac{1}{2} \ln|x^2 + 1| + \arctan x \right) + K= -\frac{1}{2} \ln|x^2 + 1| - \arctan x + K
	\]



\newpage



% 02/11
\checkin{02/11} The integral $\ds\int_1^\infty \dfrac{dx}{\sqrt[5]{x^{12}}}$ converges. \pspace

\sol The statement is \textit{true}. Recall that $\ds\int_1^\infty \dfrac{dx}{x^p}$ converges if $p > 1$ and diverges otherwise. Observe that $\ds \int_1^\infty \dfrac{dx}{\sqrt[5]{x^{12}}}= \int_1^\infty \dfrac{dx}{x^{12/5}}$. Because the power of $x$, $\frac{12}{5}$, is greater than 1, we know that this integral converges. We can see this directly:
	\[
	\ds\int_1^\infty \dfrac{dx}{\sqrt[5]{x^{12}}}:= \lim_{N \to \infty} \int_1^N \dfrac{dx}{x^{12/5}}= \lim_{N \to \infty} -\dfrac{5}{7x^{7/5}} \bigg|_1^N= \lim_{N \to \infty} -\dfrac{5}{7 N^{7/5}} - \dfrac{-5}{7(1^{7/5})}= 0 + \dfrac{5}{7}= \dfrac{5}{7}
	\]
Therefore, the integral converges to $\frac{5}{7}$. \pvspace{1.3cm}



% 02/20
\checkin{02/20} Considering the infinite series $\ds\sum_{n=1}^\infty \dfrac{n - 1}{n^3}$ using the Divergence Test, because $\ds\lim_{n \to \infty} \dfrac{n - 1}{n^3}= 0$, the series converges converges. \pspace

\sol The statement is \textit{false}. The Divergence Test can \textit{never} determine that a series converges; otherwise, they would have called it the Convergence Test. The Divergence Test states that for a series $\ds\sum_{n=1}^\infty a_n$, the series diverges if $\ds\lim_{n \to \infty} a_n \neq 0$. If $\ds\lim_{n \to \infty} a_n= 0$, we know zero information about whether the series converges or diverges. Because $\ds\lim_{n \to \infty} \dfrac{n - 1}{n^3}= 0$, we cannot yet determine whether the series $\ds\sum_{n=1}^\infty \dfrac{n - 1}{n^3}$ converges or diverges. In fact, Because for `large' $n$, $\dfrac{n - 1}{n^3} \approx \dfrac{n}{n^3}= \dfrac{1}{n^2}$ and the fact that the series $\ds\sum_{n=1}^\infty \dfrac{1}{n^2}$ converges, we suspect that the series $\sum_{n=1}^\infty \dfrac{n - 1}{n^3}$ converges. \pvspace{1.3cm}



% 02/25
\checkin{02/25} The series $\ds\sum_{n=1}^\infty 5 \left( \dfrac{2^{2n+1}}{3^n} \right)$ diverges. \pspace

\sol The statement is \textit{true}. This series is geometric, i.e. the series can be put in the form $\ds\sum a r^n$. Observe that\dots
	\[
	\sum_{n=1}^\infty 5 \left( \dfrac{2^{2n+1}}{3^n} \right)= \sum_{n=1}^\infty 5 \left( \dfrac{2^{2n} \cdot 2}{3^n} \right)= \sum_{n=1}^\infty 10 \left( \dfrac{(2^2)^n}{3^n} \right)= \sum_{n=1}^\infty 10 \left( \dfrac{2^2}{3} \right)^n= \sum_{n=1}^\infty 10 \left( \dfrac{4}{3} \right)^n
	\]
Observe that this series is geometric with $a= 10$ and $r= \frac{4}{3}$. Because $|r|= \frac{4}{3} \geq 1$, the series diverges by the Geometric Series Test. \pvspace{1.3cm}



\newpage



% 02/27
\checkin{02/27} The series $\ds\sum_{n=2}^\infty \dfrac{n^2 + 1}{\sqrt{n^3 - 5}}$ diverges. \pspace

\sol The statement is \textit{true}. Observe that for `large' $n$, $\dfrac{n^2 + 1}{\sqrt{n^3 - 5}} \approx \dfrac{n^2}{\sqrt{n^3}}= \dfrac{n^2}{n^{3/2}}= n^{1/2}$. Because the series $\ds\sum_{n=2}^\infty \sqrt{n}$ diverges (by the Divergence Test), we suspect that the series $\ds\sum_{n=2}^\infty \dfrac{n^2 + 1}{\sqrt{n^3 - 5}}$ diverges. We can prove this with the Direct Comparison Test:
	\[
	\sum_{n=2}^\infty \dfrac{n^2 + 1}{\sqrt{n^3 - 5}} > \sum_{n=2}^\infty \dfrac{n^2}{\sqrt{n^3}}= \sum_{n=2}^\infty \sqrt{n}
	\]
Because $\ds\lim_{n \to \infty} \sqrt{n}= \infty \neq 0$, the series $\ds\sum_{n=2}^\infty \sqrt{n}$ diverges by the Divergence Test. Therefore, $\sum_{n=2}^\infty \dfrac{n^2 + 1}{\sqrt{n^3 - 5}}$ diverges by the Direct Comparison Test. Alternatively, we know that because $\ds\lim_{n \to \infty} \sqrt{n}= \infty \neq 0$, the series $\ds\sum_{n=2}^\infty \sqrt{n}$ diverges by the Divergence Test. But then because\dots
	\[
	\begin{aligned}
	\lim_{n \to \infty} \dfrac{\;\;\dfrac{n^2 + 1}{\sqrt{n^3 - 5}}\;\;}{\sqrt{n}}&= \lim_{n \to \infty} \dfrac{n^2 + 1}{\sqrt{n} \cdot \sqrt{n^3 - 5}} \\
	&= \lim_{n \to \infty} \dfrac{n^2 + 1}{\sqrt{n^4 - 5n}} \\
	&= \lim_{n \to \infty} \dfrac{n^2 + 1}{\sqrt{n^4 - 5n}} \cdot \dfrac{1/n^2}{1/n^2} \\
	&= \lim_{n \to \infty} \dfrac{\;\;1 + \frac{1}{n^2}\;\;}{\dfrac{\sqrt{n^4 - 5n}}{n^2}} \\
	&= \lim_{n \to \infty} \dfrac{\;\;1 + \frac{1}{n^2}\;\;}{\dfrac{\sqrt{n^4 - 5n}}{\sqrt{n^4}}} \\
	&= \lim_{n \to \infty} \dfrac{\;\;1 + \frac{1}{n^2}\;\;}{\sqrt{\dfrac{n^4 - 5n}{n^4}}} \\
	&= \lim_{n \to \infty} \dfrac{\;\;1 + \frac{1}{n^2}\;\;}{\sqrt{1 - \frac{5}{n^3}}} \\
	&= \dfrac{1 + 0}{\sqrt{1 - 0}} \\
	&= 1 < \infty
	\end{aligned}
	\]
Because this limit is not also $0$, by the Limit Comparison Test, the series $\ds\sum_{n=2}^\infty \dfrac{n^2 + 1}{\sqrt{n^3 - 5}}$ also diverges. \pvspace{1.3cm}



% 03/04
\checkin{03/04} The series $\ds\sum_{n= -5}^\infty \dfrac{n^2 + 5}{2^n - 6}$ converges. \pspace

\sol The statement is \textit{true}. Because for `large' $n$, $\dfrac{n^2 + 5}{2^n - 6} \approx \dfrac{n^2}{2^n}$ and the fact that $2^n$ grows much faster than $n^2$, we suspect that $\ds\sum_{n=-5}^\infty \dfrac{n^2}{2^n}$ converges. Hence, we suspect $\ds\sum_{n= -5}^\infty \dfrac{n^2 + 5}{2^n - 6}$ converges. Because the series $\ds\sum_{n= -5}^\infty \dfrac{n^2 + 5}{2^n - 6}$ converges if and only if the series $\ds\sum_{n= 4}^\infty \dfrac{n^2 + 5}{2^n - 6}$ (whose terms are all positive),\footnote{We only need $n \geq 3$ for the Limit Comparison Test. However, we need $n \geq 4$ for our Limit Comparison Test.} it suffices to show that $\ds\sum_{n= 4}^\infty \dfrac{n^2 + 5}{2^n - 6}$ converges. We know\dots
	\[
	\lim_{n \to \infty} \dfrac{\;\;\dfrac{(n + 1)^2}{2^{n+1}}}{\dfrac{n^2}{2^n}}= \lim_{n \to \infty} \dfrac{(n + 1)^2}{2^{n+1}} \cdot \dfrac{2^n}{n^2}= \lim_{n \to \infty} \dfrac{(n + 1)^2}{n^2} \cdot \dfrac{2^n}{2^{n+1}}= \lim_{n \to \infty} \left( \dfrac{n + 1}{n} \right)^2 \cdot \dfrac{1}{2}= 1^2 \cdot \dfrac{1}{2}= \dfrac{1}{2} < 1
	\]
Therefore, by the Ratio Test, the series $\ds\sum_{n= 4}^\infty \dfrac{n^2}{2^n}$ converges absolutely. Alternatively, we have\dots
	\[
	\lim_{n \to \infty} \left| \dfrac{n^2}{2^n} \right|^{1/n}= \lim_{n \to \infty} \dfrac{(\sqrt[n]{n})^2}{2}= \dfrac{1^2}{2}= \dfrac{1}{2} < 1
	\]
Therefore, by the Root Test, the series $\ds\sum_{n= 4}^\infty \dfrac{n^2}{2^n}$ converges absolutely. But then\dots
	\[
	\lim_{n \to \infty} \dfrac{\;\;\dfrac{n^2 + 5}{2^n - 6}\;\;}{\dfrac{n^2}{2^n}}= \lim_{n \to \infty} \dfrac{2^n(n^2 + 5)}{n^2 (2^n - 6)}= \lim_{n \to \infty} \dfrac{2^n}{2^n - 6} \cdot \dfrac{n^2 + 5}{n^2}= 1 \cdot 1= 1 < \infty
	\]
Because this limit is also not $0$, the series $\ds\sum_{n= 4}^\infty \dfrac{n^2 + 5}{2^n - 6}$ converges by the Limit Comparison Test. Alternatively, using the fact that $2^{n - 1} > 6$ for $n \geq 4$, we have\dots
	\[
	\sum_{n= 4}^\infty \dfrac{n^2 + 5}{2^n - 6} < \sum_{n= 4}^\infty \dfrac{n^2 + 5n^2}{2^n - 2^{n-1}}= \sum_{n= 4}^\infty \dfrac{6n^2}{2^{n-1}}= 12 \sum_{n= 4}^\infty \dfrac{n^2}{2^n}
	\]
Therefore, the series $\ds\sum_{n= 4}^\infty \dfrac{n^2 + 5}{2^n - 6}$ converges by the Direct Comparison Test.



\newpage



% 03/06
\checkin{03/06} The series $\ds\sum_{n=1}^\infty \dfrac{5n - 2}{n^3 + 1}$ converges absolutely. \pspace

\sol The statement is \textit{true}. For large $n$, we have $\dfrac{5n - 2}{n^3 + 1} \approx \dfrac{5n}{n^3}= \dfrac{5}{n^2}$. Because the series $\ds\sum_{n=1}^\infty \dfrac{1}{n^2}$ converges, we suspect this series converges. First, observe that the series $\ds\sum_{n=1}^\infty \dfrac{1}{n^2}$ converges by the $p$-test with $p= 2 > 1$. But then\dots
	\[
	\lim_{n \to \infty} \dfrac{\;\;\dfrac{5n - 2}{n^3 + 1}\;\;}{\dfrac{1}{n^2}}= \lim_{n \to \infty} \dfrac{n^2(5n - 2)}{n^3 + 1}= \lim_{n \to \infty} \dfrac{5n^3 - 2n^2}{n^3 + 1}= \dfrac{5}{1}= \underbrace{5}_{\neq 0} < \infty
	\]
Therefore, by the Limit Comparison Test, the series $\ds\sum_{n=1}^\infty \dfrac{5n - 2}{n^3 + 1}$ converges. Alternatively, observe that\dots
	\[
	\sum_{n=1}^\infty \dfrac{5n - 2}{n^3 + 1} < \sum_{n=1}^\infty \dfrac{5n}{n^3}= 5 \sum_{n=1}^\infty \dfrac{1}{n^2}
	\]
Therefore, by the Direct Comparison Test, the series $\ds\sum_{n=1}^\infty \dfrac{5n - 2}{n^3 + 1}$ converges. \pspace

But all the terms of $\ds\sum_{n=1}^\infty \dfrac{5n - 2}{n^3 + 1}$ are positive. [Observe that $n^3 + 1 > 0$ for $n \geq 1$ and $5n - 2 > 0$ so long as $n > \frac{2}{5}$.] Therefore, $\ds\sum_{n=1}^\infty \dfrac{5n - 2}{n^3 + 1}$ converges absolutely. \pvspace{1.3cm}



% 03/18
\checkin{03/18} The series $\ds\sum_{n=1}^\infty \left( \dfrac{n + 1}{2n + 3} \right)^n$ converges. \pspace

\sol The statement is \textit{true}. Observe that\dots
	\[
	\lim_{n \to \infty} \sqrt[n]{|a_n|}= \lim_{n \to \infty} \left| \left( \dfrac{n + 1}{2n + 3} \right)^n \right|^{1/n}= \lim_{N \to \infty} \dfrac{n + 1}{2n + 3}= \dfrac{1}{2} < 1
	\]
Therefore, by the Root Test, the series converges absolutely. Alternatively, we can use the Ratio Test:
	\[
	\begin{aligned}
	\lim_{n \to \infty} \left| \dfrac{a_{n+1}}{a_n} \right|&= \lim_{n \to \infty} \dfrac{\;\;\left( \dfrac{(n +1) + 1}{2(n + 1)+ 3} \right)^{n+1}\;\;}{\left( \dfrac{n + 1}{2n + 3} \right)^n} \\
	&= \lim_{n \to \infty} \dfrac{\;\;\left( \dfrac{n + 2}{2n + 5} \right)^{n+1}\;\;}{\left( \dfrac{n + 1}{2n + 3} \right)^n} \\\
	&= \lim_{n \to \infty} \dfrac{\;\;\left( \dfrac{n + 2}{2n + 5} \right)^n \left( \dfrac{n + 2}{2n + 5} \right)\;\;}{\left( \dfrac{n + 1}{2n + 3} \right)^n} \\
	&= \lim_{n \to \infty} \dfrac{\;\;\left( \dfrac{n + 2}{2n + 5} \right)^n\;\;}{\left( \dfrac{n + 1}{2n + 3} \right)^n} \cdot  \dfrac{n + 2}{2n + 5} \\
	&= \left( \dfrac{\;\;\dfrac{n + 2}{2n + 5}\;\;}{\dfrac{n + 1}{2n + 3}} \right)^n \cdot \dfrac{n + 2}{2n + 5} \\
	&= \left( \dfrac{(n + 2)(2n + 3)}{(2n + 5)(n + 1)} \right)^n \cdot \dfrac{n + 2}{2n + 5} \\
	&= 1 \cdot \dfrac{1}{2} \\
	&= \dfrac{1}{2} < 1
	\end{aligned}
	\]
Therefore, the series converges absolutely.\footnote{\tiny Note. To show that $\ds\lim_{n \to \infty} \left( \dfrac{(n + 2)(2n + 3)}{(2n + 5)(n + 1)} \right)^n= 1$, let $\ds L= \lim_{n \to \infty} \left( \dfrac{(n + 2)(2n + 3)}{(2n + 5)(n + 1)} \right)^n$. But then $\ds \ln L= \lim_{n \to \infty} \ln \left( \dfrac{(n + 2)(2n + 3)}{(2n + 5)(n + 1)} \right)^n= \lim_{n \to \infty} n \ln \left( \dfrac{(n + 2)(2n + 3)}{(2n + 5)(n + 1)} \right)= \lim_{n \to \infty} \dfrac{\ln \left( \dfrac{(n + 2)(2n + 3)}{(2n + 5)(n + 1)} \right)}{1/n} \stackrel{\text{L.H.}}{=} \lim_{n \to \infty} \dfrac{n^2(4n + 7)}{4n^4 + 28n^3 + 71n^2 + 77n + 30}= 0$. But then $\ln L= 0$, so that $L= e^0= 1$.} \pvspace{1.3cm}



% 03/27
\checkin{03/27} If a power series has an interval of convergence of $(-1, 3]$, then the center is $x= 1$ and the radius of convergence is $R= 2$. \pspace

\sol The statement is \textit{true}. We know that the center of the interval must be $c= \dfrac{3 + (-1)}{2}= \dfrac{2}{2}= 1$. Therefore, the center is $x= 1$. The radius of convergence is half the width of the interval. But then the radius of convergence is $R= \dfrac{3 - (-1)}{2}= \dfrac{3 + 1}{2}= \dfrac{4}{2}= 2$. \pvspace{1.3cm}


































\end{document}