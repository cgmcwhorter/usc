\documentclass[11pt,letterpaper]{article}
\usepackage[lmargin=1in,rmargin=1in,bmargin=1in,tmargin=1in]{geometry}
\usepackage{checkins}

% -------------------
% Content
% -------------------
\begin{document}
\thispagestyle{title}

% 01/14
\checkin{01/14} The vector $\mathbf{u}= \langle 2, -1, 0 \rangle$ is a unit vector. \pspace

\sol The statement is \textit{false}. We know a unit vector has length 1. We know if $\mathbf{v}= \langle x_1, x_2, \ldots, x_n \rangle$, then $\|\mathbf{v}\|= \sqrt{x_1^2 + x_2^2 + \cdots + x_n^2}$. But then $\|\mathbf{u}\|= \| \langle 2, -1, 0 \rangle= \sqrt{2^2 + (-1)^2 + 0^2}= \sqrt{4 + 1 + 0}= \sqrt{5} \neq 1$. Therefore, $\mathbf{u}$ is not a unit vector. \pvspace{1.3cm}


% 01/16
\checkin{01/16} Suppose that $\mathbf{u}, \mathbf{v} \in \mathbb{R}^n$. If $\mathbf{u} \cdot \mathbf{v}= 0$, then either $\mathbf{u}= 0$ or $\mathbf{v}= 0$. Furthermore, $\mathbf{u} \perp \mathbf{v}$. \pspace

\sol The statement is \textit{false}. If $\mathbf{u} \cdot \mathbf{v}= 0$, it is true that $\mathbf{u}$ and $\mathbf{v}$ are perpendicular (so long as neither of them are nonzero). Furthermore, if $\mathbf{u}$ and $\mathbf{v}$ are perpendicular, then $\mathbf{u} \cdot \mathbf{v}= 0$. However, if $\mathbf{u} \cdot \mathbf{v}= 0$, it need not be the case that $\mathbf{u}$ and $\mathbf{v}$ are zero. For instance, if $\mathbf{u}= \langle 1, 0, \ldots, 0 \rangle$ and $\mathbf{v}= \langle 0, 1, 0, \ldots, 0 \rangle$, then $\mathbf{u} \cdot \mathbf{v}= 1(0) + 0(1) + 0(0) + \cdots + 0(0)= 0$ but neither $\mathbf{u}$ nor $\mathbf{v}$ are zero. Furthermore, it is impossible that $\mathbf{u}= 0$ or $\mathbf{v}= 0$. Both $\mathbf{u}, \mathbf{v}$ are \textit{vectors} while 0 is a scalar. \pvspace{1.3cm}



% 01/21
\checkin{01/21} The augmented matrices $\begin{pmatrix*}[r] 1 & 2 & 3 & 0 \\ -1 & 5 & 1 & 5 \\ 1 & -1 & 1 & -2 \end{pmatrix*}$ and $\begin{pmatrix*}[r] -1 & 5 & 1 & 5 \\ 1 & 2 & 3 & 0 \\ 2 & -2 & 2 & -4 \end{pmatrix*}$ represent equivalent systems of linear equations. \pspace

\sol The statement is \textit{true}. We know that for systems of linear equations, we can represent these systems as augmented matrices. Elementary row operations on these matrices correspond to viable arithmetic operations on the system of equation `side.' These operations preserve the solutions (if any) to the system of equations, i.e. produce equivalent systems. So, we can perform any elementary row operation on these augmented matrices and obtain equivalent systems: interchange rows, scale a row by a nonzero scalar, and replace any single row with a linear combination of other rows. Observe we can obtain the second matrix from the first by interchanging the first two rows and scaling the third row by 2. Therefore, the systems of linear equations represented by these matrices are equivalent. If the first matrix was an augmented matrix that was `untouched' from the original system. The system of equations was\dots
	\[
	\begin{cases}
	x + 2y + 3z= 0 \\
	-x + 5y + z= 5 \\
	x - y + z= -2 
	\end{cases}
	\]
which has solution $(-\frac{1}{2}, 1, -\frac{1}{2})$. \pvspace{1.3cm}



% 01/23
\checkin{01/23} There exists a linear system consisting of some number of equations with some number of variables such that there are five solutions to the given system. \pspace

\sol The statement is \textit{false}. For a system of linear equations, there are only three possibilities: there are no solutions to the system of equations, there is \textit{one} solution to the system of equations, or there are infinitely many solutions. It is not possible to have any other number of solutions. \pvspace{1.3cm}



% 01/26
\checkin{01/26} If a system of linear equations has a unique solution, then the RREF of the augmented matrix has a pivot position in every column except for the last one---which corresponds to the solution. \pspace 

\sol 

% 01/28
\checkin{01/28} Consider a matrix equation $A \mathbf{x}= \mathbf{b}$. If the RREF of $A$ has no zero rows, then the matrix-vector equation has a solution. \pspace

\sol The statement is \textit{true}. 

% 01/30
\checkin{01/30} If REF of an augmented matrix coming from a system of linear equations is $\begin{pmatrix} 1 & 1 & 1 & -1 & 0 \\ 0 & 1 & -2 & 1 & 4 \\ 0 & 0 & 0 & 2 & 6 \end{pmatrix}$, then the solution is $(x_1, x_2, x_3, x_4)= (2 - 3t, 2t + 1, t, 3)$, where $t$ is any real number. \pspace

\sol The statement is \textit{true}. 
































\end{document}