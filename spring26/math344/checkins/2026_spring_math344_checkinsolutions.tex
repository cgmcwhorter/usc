\documentclass[11pt,letterpaper]{article}
\usepackage[lmargin=1in,rmargin=1in,bmargin=1in,tmargin=1in]{geometry}
\usepackage{checkins}

% -------------------
% Content
% -------------------
\begin{document}
\thispagestyle{title}

% 01/14
\checkin{01/14} The vector $\mathbf{u}= \langle 2, -1, 0 \rangle$ is a unit vector. \pspace

\sol The statement is \textit{false}. We know a unit vector has length 1. We know if $\mathbf{v}= \langle x_1, x_2, \ldots, x_n \rangle$, then $\|\mathbf{v}\|= \sqrt{x_1^2 + x_2^2 + \cdots + x_n^2}$. But then $\|\mathbf{u}\|= \| \langle 2, -1, 0 \rangle= \sqrt{2^2 + (-1)^2 + 0^2}= \sqrt{4 + 1 + 0}= \sqrt{5} \neq 1$. Therefore, $\mathbf{u}$ is not a unit vector. \pvspace{1.3cm}


% 01/16
\checkin{01/16} Suppose that $\mathbf{u}, \mathbf{v} \in \mathbb{R}^n$. If $\mathbf{u} \cdot \mathbf{v}= 0$, then either $\mathbf{u}= 0$ or $\mathbf{v}= 0$. Furthermore, $\mathbf{u} \perp \mathbf{v}$. \pspace

\sol The statement is \textit{false}. If $\mathbf{u} \cdot \mathbf{v}= 0$, it is true that $\mathbf{u}$ and $\mathbf{v}$ are perpendicular (so long as neither of them are nonzero). Furthermore, if $\mathbf{u}$ and $\mathbf{v}$ are perpendicular, then $\mathbf{u} \cdot \mathbf{v}= 0$. However, if $\mathbf{u} \cdot \mathbf{v}= 0$, it need not be the case that $\mathbf{u}$ and $\mathbf{v}$ are zero. For instance, if $\mathbf{u}= \langle 1, 0, \ldots, 0 \rangle$ and $\mathbf{v}= \langle 0, 1, 0, \ldots, 0 \rangle$, then $\mathbf{u} \cdot \mathbf{v}= 1(0) + 0(1) + 0(0) + \cdots + 0(0)= 0$ but neither $\mathbf{u}$ nor $\mathbf{v}$ are zero. Furthermore, it is impossible that $\mathbf{u}= 0$ or $\mathbf{v}= 0$. Both $\mathbf{u}, \mathbf{v}$ are \textit{vectors} while 0 is a scalar. \pvspace{1.3cm}




























\end{document}