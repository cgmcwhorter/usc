\documentclass[11pt,letterpaper]{article}
\usepackage[lmargin=1in,rmargin=1in,bmargin=1in,tmargin=1in]{geometry}
\usepackage{checkins}

\pgfplotsset{soldot/.style={color=black,only marks,mark=*},
		holdot/.style={color=black,fill=white,only marks,mark=*},
		compat=1.12
}

\newcommand{\boxseven}[4]{%
	\draw[thick] (0,0) -- (4,0) -- (4,4) -- (0,4) -- (0,0);
	\draw[thick] (0,2) -- (4,2);
	\draw[thick] (2,0) -- (2,4);
	% '7'
	\draw[line width=0.03cm] (1.7,2.2) -- (2.3,2.2) -- (1.7,1.6);
	% Entries
	\node at (1,3) {$#1$};	% u
	\node at (3,1) {$#2$};	% dv
	\node at (1,1) {$#3$};	% du
	\node at (3,3) {$#4$};	% 
}

\usetikzlibrary{calc}
\usepackage{booktabs}
\tikzset{Arrow Style/.style={text=black, font=\boldmath}}
\newcommand{\tikzmark}[1]{%
    \tikz[overlay, remember picture, baseline] \node (#1) {};%
}
\newcommand*{\XShift}{0.5em}
\newcommand*{\YShift}{0.5ex}
\NewDocumentCommand{\DrawArrow}{s O{} m m m}{%
    \begin{tikzpicture}[overlay,remember picture]
        \draw[->, thick, Arrow Style, #2] 
                ($(#3.west)+(\XShift,\YShift)$) -- 
                ($(#4.east)+(-\XShift,\YShift)$)
        node [midway,above] {#5};
    \end{tikzpicture}%
}

% -------------------
% Content
% -------------------
\begin{document}
\thispagestyle{title}

% 01/15
\checkin{01/15} Making the $u$-substitution $u= x^2 + 1$ in $\ds\int_0^1 \dfrac{5x}{x^2 + 1}\;dx$, we obtain $\ds\int_0^1 \dfrac{5}{2u} \;du$. \pspace

\sol The statement is \textit{false}. One must always remember to change the limits when making a substitution in a definite integral. If we choose $u= x^2 + 1$, then $du= 2x \;dx$. If $x= 0$, then $u= 0^2 + 1= 1$. If $x= 1$, then $u= 1^2 + 1= 2$. But then\dots
	\[
	\int_0^1 \dfrac{5x}{x^2 + 1}\;dx= 5 \int_0^1 \dfrac{x}{x^2 + 1}\;dx= \dfrac{5}{2} \int_0^1 \dfrac{2x}{x^2 + 1}\;dx= \dfrac{5}{2} \int_1^2 \dfrac{du}{u}
	\]
We also have\dots
	\[
	\dfrac{5}{2} \int_1^2 \dfrac{du}{u}= \dfrac{5}{2}\,\ln|u| \;\bigg|_1^2= \dfrac{5}{2}\; ( \ln 2 - \ln 1)= \dfrac{5}{2}\; ( \ln 2 - 0)= \dfrac{5}{2}\,\ln(2)= \ln(\sqrt{32}) \approx 1.73287
	\] \pvspace{1.3cm}



% 01/20
\checkin{01/20} We should integrate $\ds\int x e^{x^2} \;dx$ using integration-by-parts. \pspace

\sol The statement is \textit{false}. This certainly looks like an integration-by-parts integral---specifically, a tabular integral (because it is of the form polynomial times trig). The only possible choices for $dv$ would be $e^{x^2}$, which cannot be integrated, or $x e^{x^2}$, which is the entire integral and we already `don't know' how to integrate, or $x$. This last choices forces $u= e^{x^2}$. But then the new integral produced is $\ds\int x^3 e^{x^2} \;dx$---which is even `worse.' Instead, observe that the given integral can be found by $u$-substitution using $u= x^2$, so that $du= 2x \;dx$. But then\dots
	\[
	\int x e^{x^2} \;dx= \dfrac{1}{2} \int 2x e^{x^2} \;dx= \dfrac{1}{2} \int e^u \;du= \dfrac{e^u}{2} + C= \dfrac{e^{x^2}}{2} + C
	\]
There is a heavy overlap between $u$-substitution and integration-by-parts in terms of what the integrals that require these methods `look like.' One needs to be careful when determining when to use which method. Some integrals require both methods! \pvspace{1.3cm}



% 01/22
\checkin{01/22} The integral $\ds\int x^2 3^x \;dx$ can be integrated using tabular integration. \pspace

\sol The statement is \textit{true}. We anticipate that this might be an integration-by-parts tabular integral because it is of the form polynomial times exponential. Using LIATE, we choose $u= x^2$, which forces $dv= 3^x$. Note that $\ds\int 3^x \;dx= \dfrac{3^x}{\ln 3} + C$. We then have\dots\par
	\[
	\begin{array}{c @{\hspace*{1.3cm}} c} \toprule
	u & dv \\ \cmidrule(lr){1-2}
	x^2 \tikzmark{Left 1} & \tikzmark{Right 1} 3^x \\[0.3cm]
	2x \tikzmark{Left 2} & \tikzmark{Right 2} \dfrac{3^x}{\ln 3} \\[0.3cm]
	2 \tikzmark{Left 3} & \tikzmark{Right 3} \dfrac{3^x}{(\ln 3)^2} \\[0.3cm]
	0 \tikzmark{Left 4} & \tikzmark{Right 4} \dfrac{3^x}{(\ln 3)^3} \\[0.3cm]
	
	\DrawArrow{Left 1}{Right 2}{+}
	\DrawArrow{Left 2}{Right 3}{--}
	\DrawArrow{Left 3}{Right 4}{+}
	\end{array}
	\]
Therefore, we have\dots
	\[
	\begin{aligned}
	\int x^2 3^x \;dx&= \boxed{\dfrac{x^2 (3^x)}{\ln 3} - \dfrac{2x (3^x)}{(\ln 3)^2} + \dfrac{2(3^x)}{(\ln 3)^3} + C} \\[0.3cm]
	&= \dfrac{3^x}{(\ln 3)^3} \left( x^2 (\ln 3)^2 - 2x (\ln 3) + 2 \right) + C \\[0.3cm]
	&= \dfrac{3^x \left( x^2 (\ln 3)^2 - x (2 \ln 3) + 2 \right)}{(\ln 3)^3} + C \\[0.3cm]
	&= \dfrac{3^x \left( x^2 (\ln 3)^2 - x \ln 9 + 2 \right)}{(\ln 3)^3} + C
	\end{aligned}
	\] \pvspace{1.3cm}


% 01/27
\checkin{01/27} To integrate $\ds\int \tan^2 x \sec x \;dx$, one chooses $u= \sec x$. \pspcae

\sol The statement is \textit{false}. 


% 01/29
\checkin{01/29} Consider the integral $\ds\int (x^2 + 4)^3 \;dx$. One can recognize this is a trig. sub. because the ``$a^2 + b^2= c^2$--like'' term $x^2 + 4$, and can then make the substitution $x= 2 \tan \theta$. \pspace

\sol The statement is \textit{true}. 














\end{document}