\documentclass[11pt,letterpaper]{article}
\usepackage[lmargin=1in,rmargin=1in,bmargin=1in,tmargin=1in]{geometry}
\usepackage{
	amsmath,			% Use AMS Math
	enumerate,		% Enumerate Environments
	microtype,			% Improved Typesetting
	multicol,			% Use Multiple Columns
}

% -------------------
% Fonts
% -------------------
\usepackage[T1]{fontenc}
\usepackage{charter}

\setlength{\parindent}{0ex} % Paragraph Indentation

% -------------------
% Commands
% -------------------
\newcommand\ds{\displaystyle}
\newenvironment{3enumerate}{%
	\begin{enumerate}[(1)]
	\begin{multicols}{3}
	}{%
	\end{multicols}
	\end{enumerate}
}

\newcounter{problem}
\newcommand{\prob}{\stepcounter{problem}%
\noindent\textbf{Problem \theproblem. }}

\newcommand{\pspace}{\par\vspace{\baselineskip}}

% -------------------
% Tikz/PGFPlots
% -------------------
\usepackage{pgfplots}
\pgfplotsset{compat=newest}


% -------------------
% Content
% -------------------
\begin{document}
\pagenumbering{gobble}

% Title
\begin{center} {\bfseries \LARGE $\pi$-Day \par Polar Form \& De Moivre's Theorem} \end{center} \par\vspace{0.2\baselineskip}

{\scriptsize \noindent {\bfseries \Large Quick Facts}
	\begin{itemize}
	\item $z= a + bi$ is a complex number. This form of $z$ is called the \textit{rectangular form/representation}.
	\item $|z|$ is the \textit{absolute value} or \textit{modulus} of $z$. Furthermore, $|z|= \sqrt{a^2 + b^2}$ and is the distance from the `point' $z$ to the origin. [Draw a right triangle!]
	\item The \textit{polar form/representation} of $z$ is $r (\cos \theta + i \sin \theta)$, where $r= |z|$ and $\theta$ is the `angle plotting $z$ makes.'
	\item We find the polar form of $z$ by finding $r$, finding $\theta$, and then writing out the polar form.
	\item De Moivre's (duh-mwah-vwurr) Theorem: $z^n= r^n \big(\cos(n\theta) + i \sin(n\theta) \big)$.
	\item We use De Moivre's Theorem to compute powers of $z$ by finding the polar form of $z$, writing out the expression above, and then simplifying the expression.
	\item We use the following to compute roots of complex numbers: 
		\[
		\begin{aligned}
		\underbrace{\sqrt[n]{z}}_{\text{Or } z^{1/n}}&= \sqrt[n]{r} \, \left( \cos \left( \dfrac{\theta + 2\pi k}{n} \right) + i \sin \left( \dfrac{\theta + 2\pi k}{n} \right) \right) & \qquad \text{(radians)} \\
		\underbrace{\sqrt[n]{z}}_{\text{Or } z^{1/n}}&= \sqrt[n]{r} \, \left( \cos \left( \dfrac{\theta + 360^\circ k}{n} \right) + i \sin \left( \dfrac{\theta + 360^\circ k}{n} \right) \right) & \text{(degrees)}
		\end{aligned}
		\]
	where $k= 0, 1, \ldots, n-1$. These roots are equally spaced points (separated by an angle $\frac{360^\circ}{n}$, starting at angle $\frac{\theta}{n}$) on the circle at the origin with radius $\sqrt[n]{r}$.
	\item To find roots of complex numbers, we find the polar form of $z$, use the above expression (writing it out for each of $k= 1, 2, \ldots, n - 1$, and then simplify each expression.
	\end{itemize}
}

\noindent\rule{\textwidth}{0.4pt}

\end{document}