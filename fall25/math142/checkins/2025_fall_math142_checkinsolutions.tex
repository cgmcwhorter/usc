\documentclass[11pt,letterpaper]{article}
\usepackage[lmargin=1in,rmargin=1in,bmargin=1in,tmargin=1in]{geometry}
\usepackage{checkins}

\pgfplotsset{soldot/.style={color=black,only marks,mark=*},
		holdot/.style={color=black,fill=white,only marks,mark=*},
		compat=1.12
}

% -------------------
% Content
% -------------------
\begin{document}
\thispagestyle{title}

% 08/21
\checkin{08/21} The integral $\ds\int x \sqrt[3]{x - 2} \;dx$ can be treated as a `shifting integral' by using the $u$-substitution $u= x - 2$. \pspace

\sol The statement is \textit{true}. We `want' to be able to distribute the $x$ across the cube-root but we cannot---this is not a valid operation. However, if we make the $u$-substitution $u= x - 2$, then we will be able to distribute in a way that makes this integral `routine.' So, let $u= x - 2$, then $du= dx$. Moreover, because $u= x - 2$, we know that $x= u + 2$. But then\dots
	\[
	\int x \sqrt[3]{x - 2} \;dx= \int (u + 2) \sqrt[3]{u} \;du= \int \left( u^{4/3} + 2 u^{1/3} \right) \;du= \tfrac{3}{7}\, u^{7/3} + \tfrac{3}{4} \cdot 2u^{4/3} + C= \tfrac{3}{7} (x - 2)^{7/3} + \tfrac{3}{2} (x - 2)^{4/3} + C
	\]
Note that a computer algebra system may write the answer (though you will \textit{not} be expected to) like this:
	\[
	\tfrac{3}{7} (x - 2)^{7/3} + \tfrac{3}{2} (x - 2)^{4/3} + C= (x - 2)^{4/3} \left( \tfrac{3}{7} (x - 2) + \tfrac{3}{2} \right) + C= (x - 2)^{4/3} \left( \tfrac{3}{7}x + \tfrac{9}{14} \right) + C= \tfrac{3}{14} (x - 2)^{4/3} \left( 2x + 3 \right) + C
	\] \pvspace{1.3cm}






























\end{document}