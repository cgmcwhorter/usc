\documentclass[11pt,letterpaper]{article}
\usepackage[lmargin=1in,rmargin=1in,bmargin=1in,tmargin=1in]{geometry}
\usepackage{style}

\setlength{\parindent}{0ex}

\usepackage{tikzsymbols}

% -------------------
% Content
% -------------------
\begin{document}

% Title
\begin{center} {\bfseries \LARGE MATH 142 --- Comment Card Responses --- Fall 2025} \end{center}

Here are select responses to comment cards given at the end of class. When computed, the class rating (out of 10) is given. The comment (often paraphrased for clarity) is given in italics/bold with the response following the comment. The responses are labeled by class date---with the class topic given. The classes are given in reverse-chronological order for ease of access to the most recent class. You may also click any of the hyperlinks below to jump to that date.

\begin{itemize}
\item \hyperref[09-04]{09/04, Thursday: Partial Fractions}
\item \hyperref[09-02]{09/02, Tuesday: Trig. Substitution}
\item \hyperref[08-28]{08/28, Thursday: Trigonometric Integrals}
\item \hyperref[08-26]{08/26, Tuesday: Integration-by-Parts}
\item \hyperref[08-21]{08/21, Thursday: Integration-by-Parts}
\item \hyperref[08-19]{08/19, Tuesday: $u$-Substitution Review}
\end{itemize}

% 09/04, Thursday: Partial Fractions
\newpage
\section*{09/04, Thursday: Partial Fractions\label{09-04}}

\begin{itemize}
\item Class Rating: 8.58/10
\item {\bfseries\itshape Lots of examples was really helpful:} I am glad that you found things helpful! Remember, you can always start the homework right away to make sure that things clicked for you. 

\item {\bfseries\itshape All the examples were helpful:}
\item {\bfseries\itshape Very good examples:}
\item {\bfseries\itshape Not explaining steps, talking at us rather than teaching:}
\item {\bfseries\itshape The pre-lecture was helpful:} 
\item {\bfseries\itshape Good job:}
\item {\bfseries\itshape Why not just teach us the shortcut?:} 
\item {\bfseries\itshape Really clear, thank you!:}


\item {\bfseries\itshape Culver's is the best fast food:}
\end{itemize}

% 09/02, Tuesday: Trig. Substitution
\newpage
\section*{09/02, Tuesday: Trig. Substitution\label{09-02}}

\begin{itemize}
\item Class Rating: 8.54/10
\item {\bfseries\itshape More clear than previous lectures but go too fast when you get on a role and explain too many steps at a time:}
\item {\bfseries\itshape The method for trig sub. was very simple and well explained:}
\item {\bfseries\itshape How and why are certain trig. functions are chosen over others?:}
\item {\bfseries\itshape Good job:}
\item {\bfseries\itshape This was better than the textbook method:}
\item {\bfseries\itshape The triangle method and examples were helpful:}
\item {\bfseries\itshape When writing tips/words say what you write as you write it:} 
\item {\bfseries\itshape I do not know when you apply each method:} 
\item {\bfseries\itshape Comparing everything to triangles made sense:} 
\end{itemize}

% 08/28, Thursday: Trigonometric Integrals
\newpage
\section*{08/28, Thursday: Trigonometric Integrals\label{08-28}}

\begin{itemize}
\item Class Rating: 8.77/10
\item {\bfseries\itshape Could you post the print out on Blackboard prior to class. The ones you hand out before class. So we can all draw on it on our iPads/computers:} 
\item {\bfseries\itshape The derivative trick for the trig. functions was helpful:}
\item {\bfseries\itshape I was sleepy:}
\item {\bfseries\itshape Please continue with the slower pacing with interactive lecture. Thumbs up:}
\item {\bfseries\itshape Was lit:}
\item {\bfseries\itshape Working through the problems set by step was good. It really helped a lot:} 
\item {\bfseries\itshape Solid class 10/10. Everything was explained and worked out well:}
\item {\bfseries\itshape Was easier than recent lessons:} 
\item {\bfseries\itshape It was hard to see certain exponents:} 
\item {\bfseries\itshape Not saying what you are using from step to step, such as power rule, makes lectures confusing, jump from step to step, not giving any formulas either really puts too much stress. Expecting us to memorize integral of everything is crazy:} 
\item {\bfseries\itshape It would be nice to have some sort of outline of what we learn in the lecture:} 
\end{itemize}

% 08/26, Tuesday: Integration-by-Parts
\newpage
\section*{08/26, Tuesday: Integration-by-Parts\label{08-26}}

\begin{itemize}
\item Class Rating: 8.29/10
\item {\bfseries\itshape The slow and concise examples were nice:}
\item {\bfseries\itshape Note unhelpful just wondering instances where on a looping integral you have two negatives making the original term:} 
\item {\bfseries\itshape Is $\ds \int u \;dv= uv \pm ab + \int u \;dv$?:}
\item {\bfseries\itshape What website/channels are good for outside help? It takes me a little longer than others to grasp the concept of things and get it:} 
\item {\bfseries\itshape Didn't realize we weren't done with IBP notes and was crashing out over the homework:} 
\item {\bfseries\itshape I appreciated the slow paced examples and explanations\dots:} 
\item {\bfseries\itshape Tabular integration was only briefly explained:} 
\item {\bfseries\itshape This was easy to understand:}
\item {\bfseries\itshape Can you go a bit slower and write down what you did instead of doing it in your head and writing it. Like how did you get from $\ds -\tfrac{\ln(2x)}{x} + \int \tfrac{1}{x^2} \;dx$ to $-\tfrac{\ln(2x)}{x} - \tfrac{1}{x} + C$:} 
\item {\bfseries\itshape Why do you set the looping ones equal to the original integral?:} 
\item {\bfseries\itshape Explain each part, we are in summer mode you tend to skip \underline{important} small steps out of habit:} 
\item {\bfseries\itshape I don't even know, I'm kind of just confused on everything:} 
\item {\bfseries\itshape Thanks for making things simple:}
\item {\bfseries\itshape The tabular/looping integrals were well explained:}
\item {\bfseries\itshape I felt that we jumped into a topic that we didn't even start the class before:}
\item {\bfseries\itshape Good job:} 
\item {\bfseries\itshape Your handwriting can be confusing:}
\item {\bfseries\itshape It was helpful to take time to explain when to use what:} 
\item {\bfseries\itshape I think you explained it well and slower:} 
\item {\bfseries\itshape Why do you think One Piece is bad (don't say pacing)?:}
\end{itemize}

% 08/21, Thursday: Integration-by-Parts
\newpage
\section*{08/21, Thursday: Integration-by-Parts\label{08-21}}

\begin{itemize}
\item {\bfseries\itshape Class Rating:} 8.45/10

\item {\bfseries\itshape It was nice to be shown examples and then to try ourselves:} I am glad that this lecture clicked more for you. I honestly try to build in as much `try it yourself' time as possible. But some lectures will have more/less of that depending on the topic. The real `do problems' time is recitation. 

\item {\bfseries\itshape The box method was helpful:} Glad you liked it. But even if it felt like it `clicked', always remember to get to the homework asap to see if it actually did. If it didn't, then you have a chance to put things on track before we dive further. Remember, it's easier (but not necessarily easy) to keep up than to catch up!

\item {\bfseries\itshape Towards the end it got complicated:} That was actually kind of the point! Those last two examples---$\ds\int x^2 e^x \;dx$ and $\ds\int e^x \sin x \;dx$---were definitely harder than the others. For the former, we had to use integration-by-parts twice. For the second, it somehow `looped' back on itself. They were awful! We did need to see the `long way'. Finding ways of avoiding having to do these integrals this way would be great\dots and is the point of the next lecture! 

\item {\bfseries\itshape It was a lot to absorb all at once:} Some lectures will be like that. Not everything is going to come right away; that's fine and natural. Remember, you can rewatch the lecture and pause at your own pace. You can also stop by to ask me or the TA questions. You can also ask the SI to help clarify points!

\item {\bfseries\itshape Have a hard time seeing the problems:} I will try to write larger and/or zoom in more!

\item {\bfseries\itshape The pacing was better today:} Great that things went smoother for you! But remember, it is always easier to follow in lecture than it is 100\% own your own. So, try the homework to see what `settled in' and what you might need to look at more!

\item {\bfseries\itshape It would be nice to know what you expect from us when solving a problem. Like the `7' method, it was kind of very quickly explained yet then we had to do a problem:} Calculus~II is all about doing problems! Though it seems like some kind of shortcut (and it is), everything I showed was everything you need to show for the `7'-method---no more or less. So, if you did those problems as I did in-class on the exam, you'd be perfectly fine!

\item {\bfseries\itshape love the 7:} It always felt like a casino type deal---777. 

\item {\bfseries\itshape I need more clarification on when to use $u$-sub. and when to use IBP:} That \textit{is} the hard part. Both $u$-substitution and integration-by-parts are \textit{very} general methods that apply to many types of integrals. The eventual goal is to be able to look at an integral and have a `feel' for what method might be best. Eventually, we will have methods that apply to specific types of integrals. So, once we have those, it is easier to work from these more specific methods `backwards' to the most general in order to figure out what approach might be best. Indeed, it is not always easy to see which types of integrals `require' $u$-substitution or integration-by-parts. But with enough problems under your belt, you will get a feel for this!

\item {\bfseries\itshape Super cool stuff:} Just wait to see the other tricks that we get to see!

\item {\bfseries\itshape Please zoom in and write bigger please. Find a way to have the iPad fill screen:} I shall try. It is a battle. If you zoom in a lot, things are bigger but then you have to move about more and not everything is on the screen. It's then easy for students to fall behind or miss writing things that are no longer on the screen. It's a balancing act and I shall do my best. 

\item {\bfseries\itshape Solid!:} Danke!

\item {\bfseries\itshape It was helpful to have some problems to do:} And there's lots more to be seen on the homework \Laughey
\end{itemize}

% 08/19, Tuesday: u-Substitution Review
\newpage
\section*{08/19, Tuesday: $u$-Substitution Review\label{08-19}}

\begin{itemize}
\item {\bfseries\itshape Class Rating:} 8.51/10

\item {\bfseries\itshape The 2nd type of $u$-sub was least helpful:} I think you mean the `hoity-toity'/`in-place' method compared to just substituting. While it can definitely feel `weirder' than the approach you might have been taught at first, this `in-place' approach is actually easier for certain types of integrals, e.g. trigonometric integrals, that we will see later. So, it is important that we get a feel for it now to make things easier for us down the line. But for sure, it takes some practice. 

\item {\bfseries\itshape We moved very fast:} We will have to move faster in some classes than others to cover everything. So, some classes will be slower or have more time for some in-class practice. Of course, today we moved a bit faster because---in theory---everything from today should have been review! If you couldn't quite remember everything, that's perfectly fine! You just came from a long summer. It's perfectly natural to have forgotten some things. But hopefully, after covering so much today, you have a feel for what you remember and what you need to review. 

\item {\bfseries\itshape Most helpful was the linear $u$-sub.:} I am glad that you found it helpful! Remember, you will want to practice this until you get a feel for it enough to be able to do it in your head. This makes the future integrals \textit{much} easier to deal with. 

\item {\bfseries\itshape Good job:} Thanks! 

\item {\bfseries\itshape Would you recommend Pearson+?:} I do not know anything about Pearson+. I cannot imagine it would have more resources than what you could find on Blackboard per topic. So, I would recommend whatever is the cheapest option for Pearson. 

\item {\bfseries\itshape It would help if class notes could be online so I can use my iPad:} If you mean the note handouts I give you, they are already on Blackboard before the class begins! If you mean the in-class notes, those I am not able to put out before the class starts. 

\item {\bfseries\itshape I accidentally sat in on the wrong class :( :} Oof. You didn't need to stay in a class that you aren't enrolled in! It's easy to get lost the first day. Definitely, don't stay in a class you aren't in. It's no problem to get up and find your actual class. You don't want to miss the first day! \dots but then I realize you'll never read this\dots oh.

\item {\bfseries\itshape You explained too much and too little:} I will do my best to try to `Goldilocks' that next time. 

\item {\bfseries\itshape Wish we did the syllabus:} We did! I covered it in a syllabus video posted to Blackboard and asked if there were any questions on it! This saved us a full class, which we can then spend to give us an extra lecture on one of the future integration techniques. I think that in the long-run, you'll much prefer this over talking about the syllabus! 

\item {\bfseries\itshape Have trouble seeing what's written:} I will do my best to write larger/zoom in more.

\item {\bfseries\itshape The note packet was really helpful:} Thanks! Remember, if you ever lose it or need another copy, everything is linked in Blackboard. So, there is a master set of notes you can print (full-size even) the notes that I handed out. 
\end{itemize}

\end{document}