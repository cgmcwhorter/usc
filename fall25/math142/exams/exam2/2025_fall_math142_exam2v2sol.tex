\documentclass[12pt,letterpaper]{exam}
\usepackage[lmargin=1in,rmargin=1in,tmargin=1in,bmargin=1in]{geometry}
\usepackage{../style/exams}

% -------------------
% Course & Exam Information
% -------------------
\newcommand{\course}{MATH 142: Exam 2}
\newcommand{\term}{Fall ---\textsubscript{\textsubscript{2}} 2025}
\newcommand{\examdate}{10/23/2025}
\newcommand{\timelimit}{75 Minutes}

\setbool{hideans}{false} % Student: True; Instructor: False
\usepackage{cancel}

% -------------------
% Content
% -------------------
\begin{document}

\examtitle
\instructions{Write your name on the appropriate line on the exam cover sheet. This exam contains \numpages\ pages (including this cover page) and \numquestions\ questions. Check that you have every page of the exam. Answer the questions in the spaces provided on the question sheets. Be sure to answer every part of each question and show all your work. If you run out of room for an answer, continue on the back of the page --- being sure to indicate the problem number.
} 
\scores
\bottomline
\newpage


% -------------------
% Questions
% -------------------
\begin{questions}

% Question 1
\newpage
\question[20] Showing all your work and fully justifying your reasoning, determine whether the following series is divergent, conditionally convergent, or absolutely convergent. 
	\[
	\sum_{n=1}^\infty (-1)^n \;\dfrac{\sqrt{n}}{n + 1}
	\] \par\vspace{0.1cm}

\vsol{\itshape\small This is an alternating series, so we apply the Alternating Series Test. We need to check two conditions for the test:
	\begin{itemize}
	\item $\ds\lim_{n \to \infty} \dfrac{\sqrt{n}}{n + 1}= 0$: This follows from a routine computation:
		\[
		\lim_{n \to \infty} \dfrac{\sqrt{n}}{n + 1} \stackrel{\text{L.H.}}{=} \lim_{n \to \infty} \dfrac{\tfrac{1}{2\sqrt{n}}}{1}= \lim_{n \to \infty} \dfrac{1}{2\sqrt{n}}= 0
		\]
	\item The sequence $\left\{ \dfrac{\sqrt{n}}{n + 1} \right\}$ is decreasing: Observe that $\tfrac{d}{dn} \left( \tfrac{\sqrt{n}}{n + 1} \right)= \tfrac{\tfrac{1}{2\sqrt{n}} (n + 1) - 1 \cdot \sqrt{n}}{(n + 1)^2}= \tfrac{1 - n}{2 \sqrt{n} (n + 1)^2}$. The denominator is always positive and $1 - n < 0$ for $n > 1$. Therefore, the sequence is decreasing. Alternatively, if we let $a_n= \tfrac{\sqrt{n}}{n + 1}$, we need to show that $a_{n+1} < a_n$. Observe that for $n \geq 1$, we have $n^2 \geq 1$ and $n \geq 1$, so that $n^2 + n \geq 2$. But then $n^2 + n - 1 \geq 1 > 0$. But $n^2 + n - 1= n(n + 2)^2 - (n + 1)^3$, so that $n(n + 2)^2 - (n + 1)^3 > 0$. Therefore, $(n + 1)^3 < n(n + 2)^2$, which dividing both side by $(n + 1)^2 (n + 2)^2$ is equivalent to $\tfrac{n + 1}{(n + 2)^2} < \tfrac{n}{(n + 1)^2}$. Taking the square root, this is $\tfrac{\sqrt{n + 1}}{n + 2} < \tfrac{\sqrt{n}}{n + 1}$, i.e. $\tfrac{\sqrt{n + 1}}{(n + 1) + 1} < \tfrac{\sqrt{n}}{n + 1}$. But this is precisely $a_{n+1} < a_n$.
	\end{itemize}
Therefore, by the Alternating Series Test, the series $\ds\sum_{n=1}^\infty (-1)^n \;\dfrac{\sqrt{n}}{n + 1}$ converges. \par\vspace{0.1cm}

To determine whether the given series is absolutely convergent, we now need to determine whether the series $\ds\sum_{n=1}^\infty \dfrac{\sqrt{n}}{n + 1}$ converges. Observe that for `large' $n$, $\dfrac{\sqrt{n}}{n + 1} \approx \dfrac{\sqrt{n}}{n}= \dfrac{1}{\sqrt{n}}$, whose infinite series diverges by the $p$-test. We then predict that the series $\ds\sum_{n=1}^\infty \dfrac{\sqrt{n}}{n + 1}$ diverges. We know the series $\ds\sum_{n=1}^\infty \dfrac{1}{\sqrt{n}}$ diverges by the $p$-test with $p= \tfrac{1}{2}$. \par\vspace{0.1cm}

Using the Limit Comparison Test, we limit compare to $\tfrac{1}{\sqrt{n}}$:
	\[
	\lim_{n \to \infty} \dfrac{\;\;\dfrac{\sqrt{n}}{n + 1}}{\dfrac{1}{\sqrt{n}}}= \lim_{n \to \infty} \dfrac{\sqrt{n}}{n + 1} \cdot \dfrac{\sqrt{n}}{1}= \lim_{n \to \infty} \dfrac{n}{n + 1} \stackrel{\text{L.H.}}{=} \lim_{n \to \infty} \dfrac{1}{1}= 1
	\]
By the Limit Comparison Test because this limit is finite and nonzero, the series $\ds\sum_{n=1}^\infty \dfrac{\sqrt{n}}{n + 1}$ has the same behavior as $\ds\sum_{n=1}^\infty \dfrac{1}{\sqrt{n}}$, so that the series $\ds\sum_{n=1}^\infty \dfrac{\sqrt{n}}{n + 1}$ diverges.}

\vsol{
\newpage
{
\thispagestyle{empty}
\itshape \small
Alternatively, we can use the Direct Comparison Test. We have\dots
	\[
	\sum_{n=1}^\infty \dfrac{\sqrt{n}}{n + 1} > \sum_{n=1}^\infty \dfrac{\sqrt{n}}{n + n}= \sum_{n=1}^\infty \dfrac{\sqrt{n}}{2n}= \dfrac{1}{2} \sum_{n=1}^\infty \dfrac{1}{\sqrt{n}}
	\]
By the Direct Comparison Test because the series $\ds\sum_{n=1}^\infty \dfrac{1}{\sqrt{n}}$ diverges, the series $\ds\sum_{n=1}^\infty \dfrac{\sqrt{n}}{n + 1}$ diverges. \par\vspace{0.1cm}

Alternatively, we can use the Integral Test. Let $f(x)= \dfrac{\sqrt{x}}{x + 1}$. We know that $f(x) > 0$ because $\sqrt{x} > 0$ and $x + 1 > 0$ for $x \geq 1$. We have already shown that $f(x)$ is decreasing. Finally, we know that $f(x)$ is continuous on $[1, \infty)$. Therefore, by the Integral Test, $\ds\sum_{n=1}^\infty f(n)$ converges if and only if $\ds\int_1^\infty f(x) \;dx$ converges. Let $u= \sqrt{x}$, so that $du= \tfrac{1}{2\sqrt{x}} \;dx$. But then $2 \sqrt{x} \,du= dx$, i.e. $2u \,du= dx$. Finally, if $u= \sqrt{x}$, then $x= u^2$. But then\dots
	\[
	\int \dfrac{\sqrt{x}}{x + 1} \;dx= \int \dfrac{u}{u^2 + 1} \cdot 2u \;du= 2\int \dfrac{u^2}{u^2 + 1} \;du= 2 \int \dfrac{(u^2 + 1) - 1}{u^2 + 1} \;du= 2 \int \left(1 - \dfrac{1}{u^2 + 1} \right) \;du
	\]
But we have\dots
	\[
	2 \int \left(1 - \dfrac{1}{u^2 + 1} \right) \;du= 2 \left(u - \arctan(u) \right) + C= 2 \left( \sqrt{x} - \arctan(\sqrt{x}) \right) + C
	\]
Therefore, we have\dots
	\[
	\begin{aligned}
	\int_1^\infty \dfrac{\sqrt{x}}{x + 1} \;dx&:=. \lim_{b \to \infty} \int_1^b \dfrac{\sqrt{x}}{x + 1} \;dx \\
	&= \lim_{b \to \infty} 2 \left( \sqrt{x} - \arctan(\sqrt{x}) \right) \bigg|_1^b \\
	&= \lim_{b \to \infty} 2 \left( \sqrt{b} - \arctan(\sqrt{b}) \right) - 2 \left( \sqrt{1} - \arctan(\sqrt{1}) \right) \\
	&= \lim_{b \to \infty} 2 ( \underbrace{\sqrt{b}}_{\to \infty} - \underbrace{\arctan(\sqrt{b})}_{\to \tfrac{\pi}{2}} ) - 2 \left( 1 - \tfrac{\pi}{4} \right) \\
	&= \infty
	\end{aligned}
	\]
Because the integral diverges, by the Integral Test, the series $\ds\sum_{n=1}^\infty \dfrac{\sqrt{n}}{n + 1}$ diverges. \par\vspace{0.1cm}

Therefore, because the series $\ds\sum_{n=1}^\infty (-1)^n \;\dfrac{\sqrt{n}}{n + 1}$ converges and the series $\ds\sum_{n=1}^\infty \dfrac{\sqrt{n}}{n + 1}$ diverges, the series $\ds\sum_{n=1}^\infty (-1)^n \;\dfrac{\sqrt{n}}{n + 1}$ is conditionally convergent.
\setcounter{page}{2}
}
}



% Question 2
\newpage
\question[10] Showing all your work and fully justifying your reasoning, determine whether the following series diverges or converges. If the series converges, find its sum:
	\[
	\sum_{n=1}^\infty \dfrac{(-2)^{n+1}}{3^{2n}}
	\] \pspace

\vsol{\itshape Observe that\dots
	\[
	\sum_{n=1}^\infty \dfrac{(-2)^{n+1}}{3^{2n}}= \sum_{n=1}^\infty \dfrac{(-2)^n (-2)^1}{(3^2)^n}= \sum_{n=1}^\infty -2 \, \dfrac{(-2)^n}{9^n}= \sum_{n=1}^\infty -2\, \left( \dfrac{-2}{9} \right)^n
	\]
Because this series has the form $\sum ar^n$ with $a= -2$ and $r= -\frac{2}{9}$, the series is geometric. Because $|r|= \tfrac{2}{9} < 1$, the series $\ds\sum_{n=1}^\infty \dfrac{(-2)^{n+1}}{3^{2n}}$ converges by the Geometric Series Test. \pspace

We know that the sum of the series is $\dfrac{\;\;\text{first term}\;\;}{1 - r}$. But then\dots
	\[
	\sum_{n=1}^\infty \dfrac{(-2)^{n+1}}{3^{2n}}= \dfrac{\;\;\dfrac{(-2)^{1+1}}{3^{2(1)}}\;\;}{1 - \left( -\dfrac{2}{9} \right)}= \dfrac{4/9}{1 + \dfrac{2}{9}}= \dfrac{4/9}{11/9}= \dfrac{4}{\cancel{9}} \cdot \dfrac{\cancel{9}}{11}= \dfrac{4}{11}
	\]
}



% Question 3
\newpage
\question[10] Showing all your work and fully justifying your reasoning, determine whether the following series converges or diverges:
	\[
	\sum_{n=1}^\infty \dfrac{n + 5}{\sqrt{9n^4 - 3}}
	\] \pspace

\vsol{\itshape \small For `large' $n$, we know that $\dfrac{n + 5}{\sqrt{9n^4 - 3}} \approx \dfrac{n}{\sqrt{9n^4}}= \dfrac{n}{3n^2}= \dfrac{1}{3} \cdot \dfrac{1}{n}$, whose infinite series diverges by the $p$-test. Therefore, we predict the series $\ds\sum_{n=1}^\infty \dfrac{n + 5}{\sqrt{9n^4 - 3}}$ diverges. We know that the series $\ds\sum_{n=1}^\infty \dfrac{1}{n}$ (the Harmonic series) diverges by the $p$-test with $p= 1$. \pspace

Using the Limit Comparison Test, we limit compare to $\dfrac{1}{n}$. We have\dots
	\[
	\begin{aligned}
	\lim_{n \to \infty} \dfrac{\;\;\dfrac{n + 5}{\sqrt{9n^4 - 3}}\;\;}{\dfrac{1}{n}}&= \lim_{n \to \infty} \dfrac{n + 5}{\sqrt{9n^4 - 3}} \cdot \dfrac{n}{1} \\
	&= \lim_{n \to \infty} \dfrac{n^2 + 5n}{\sqrt{9n^4 - 3}} \cdot \dfrac{1/\sqrt{n^4}}{1/\sqrt{n^4}} \\
	&= \lim_{n \to \infty} \dfrac{\tfrac{n^2}{n^2} + \tfrac{5n}{n^2}}{\sqrt{\dfrac{9n^4 - 3}{n^4}}} \\
	&= \lim_{n \to \infty} \dfrac{1 + \tfrac{5}{n}}{\sqrt{9 - \tfrac{3}{n^4}}} \\
	&= \dfrac{1 + 0}{\sqrt{9 - 0}} \\
	&= \dfrac{1}{3}
	\end{aligned}
	\]
By the Limit Comparison Test because this limit is finite and nonzero, the series $\ds\sum_{n=1}^\infty \dfrac{n + 5}{\sqrt{9n^4 - 3}}$ has the same behavior as $\ds\sum_{n=1}^\infty \dfrac{1}{n}$, so that the series $\ds\sum_{n=1}^\infty \dfrac{n + 5}{\sqrt{9n^4 - 3}}$ diverges. \pspace

Alternatively, we can use the Direct Comparison Test. We have\dots
	\[
	\sum_{n=1}^\infty \dfrac{n + 5}{\sqrt{9n^4 - 3}} > \sum_{n=1}^\infty \dfrac{n}{\sqrt{9n^4}}= \sum_{n=1}^\infty \dfrac{n}{3n^2}= \dfrac{1}{3} \sum_{n=1}^\infty \dfrac{1}{n}
	\]
By the Direct Comparison Test because $\ds\sum_{n=1}^\infty \dfrac{1}{n}$ diverges, the series $\ds\sum_{n=1}^\infty \dfrac{n + 5}{\sqrt{9n^4 - 3}}$ diverges.}



% Question 4
\newpage
\question[10] Showing all your work and fully justifying your reasoning, determine whether the following series converges or diverges:
	\[
	\sum_{n=1}^\infty \left( \dfrac{4n^2 + 7}{3n(n + 1)} \right)^n
	\] \pspace

\vsol{\itshape We use the Root Test:
	\[
	\hspace{-1cm} L:= \lim_{n \to \infty} \left| \left( \dfrac{4n^2 + 7}{3n(n + 1)} \right)^n \right|^{1/n}= \lim_{n \to \infty} \dfrac{4n^2 + 7}{3n(n + 1)}= \lim_{n \to \infty} \dfrac{4n^2 + 7}{3n^2 + 3n} \stackrel{\text{L.H.}}{=} \lim_{n \to \infty} \dfrac{8n}{6n + 3} \stackrel{\text{L.H.}}{=} \lim_{n \to \infty} \dfrac{8}{6}= \dfrac{4}{3} > 1
	\]
Because $L > 1$, the series $\ds\sum_{n=1}^\infty \left( \dfrac{4n^2 + 7}{3n(n + 1)} \right)^n$ diverges by the Root Test.}



% Question 5
\newpage
\question[10] Showing all your work and fully justifying your reasoning, determine whether the following series converges or diverges:
	\[
	\sum_{n=1}^\infty \dfrac{n^2 + 3n - 1}{5n^4 - 3n + 2}
	\] \pspace

\vsol{\itshape For `large' $n$, we know that $\dfrac{n^2 + 3n - 1}{5n^4 - 3n + 2} \approx \dfrac{n^2}{5n^4}= \dfrac{1}{5} \cdot \dfrac{1}{n^2}$, whose infinite series converges by the $p$-test. Therefore, we predict that $\ds\sum_{n=1}^\infty \dfrac{n^2 + 3n - 1}{5n^4 - 3n + 2}$ converges. We know that the series $\ds\sum_{n=1}^\infty \dfrac{1}{n^2}$ converges by the $p$-test with $p= 2$. [In fact, $\ds\sum_{n=1}^\infty \dfrac{1}{n^2}= \dfrac{\pi^2}{6}$.] \pspace

Using the Limit Comparison Test, we limit compare to $\dfrac{1}{n^2}$. We have\dots
	\[
	\lim_{n \to \infty} \dfrac{\;\;\dfrac{n^2 + 3n - 1}{5n^4 - 3n + 2}\;\;}{\dfrac{1}{n^2}}= \lim_{n \to \infty} \dfrac{n^2 + 3n - 1}{5n^4 - 3n + 2} \cdot \dfrac{n^2}{1}= \lim_{n \to \infty} \dfrac{n^4 + 3n^2 - n^2}{5n^4 - 3n + 2}= \dfrac{1}{5}
	\]
By the Limit Comparison Test because this limit is finite and nonzero, the series $\ds\sum_{n=1}^\infty \dfrac{n^2 + 3n - 1}{5n^4 - 3n + 2}$ has the same behavior as $\ds\sum_{n=1}^\infty \dfrac{1}{n^2}$, so that the series $\ds\sum_{n=1}^\infty \dfrac{n^2 + 3n - 1}{5n^4 - 3n + 2}$ converges. \pspace

Alternatively, we can use the Direct Comparison Test. We have\dots
	\[
	\sum_{n=1}^\infty \dfrac{n^2 + 3n - 1}{5n^4 - 3n + 2} < \sum_{n=1}^\infty \dfrac{n^2 + 3n^2}{5n^4 - 3n^4}= \sum_{n=1}^\infty \dfrac{4n^2}{2n^4}= 2 \sum_{n=1}^\infty \dfrac{1}{n^2}
	\]
By the Direct Comparison Test because the series $\ds\sum_{n=1}^\infty \dfrac{1}{n^2}$ converges, the series $\ds\sum_{n=1}^\infty \dfrac{n^2 + 3n - 1}{5n^4 - 3n + 2}$ converges.}



% Question 6
\newpage
\question[10] Showing all your work and fully justifying your reasoning, determine whether the following series converges or diverges:
	\[
	\sum_{n=0}^\infty \dfrac{n\, 3^n}{(2n)!}
	\] \pspace

\vsol{\itshape We use the Ratio Test:
	\[
	\begin{aligned}
	L:= \lim_{n \to \infty} \dfrac{\;\;\dfrac{(n+1)\, 3^{n + 1}}{\big(2(n + 1) \big)!}\;\;}{\dfrac{n\, 3^n}{(2n)!}}&= \lim_{n \to \infty} \dfrac{(n+1) 2^{n + 1}}{\big(2(n + 1) \big)!} \cdot \dfrac{(2n)!}{n\, 2^n} \\[0.3cm]
	&= \lim_{n \to \infty} \dfrac{(n+1)\, 3^{n + 1}}{(2n + 2)!} \cdot \dfrac{(2n)!}{n\, 3^n} \\[0.3cm]
	&= \lim_{n \to \infty} \dfrac{n + 1}{n} \cdot \dfrac{3^{n+1}}{3^n} \cdot \dfrac{(2n)!}{(2n + 2)!} \\[0.3cm]
	&= \lim_{n \to \infty} \dfrac{n + 1}{n} \cdot \dfrac{3^n \, 3}{3^n} \cdot \dfrac{(2n)!}{(2n + 2)(2n + 1)(2n)!} \\[0.3cm]
	&= \lim_{n \to \infty} \dfrac{n + 1}{n} \cdot \dfrac{\cancel{3^n} \, 3}{\cancel{3^n}} \cdot \dfrac{\cancel{(2n)!}}{(2n + 2)(2n + 1)\cancel{(2n)!}} \\[0.3cm]
	&= \lim_{n \to \infty} \dfrac{n + 1}{n} \cdot 3 \cdot \dfrac{1}{(2n + 2)(2n + 1)} \\[0.3cm]
	&= 1 \cdot 3 \cdot 0 \\[0.3cm]
	&= 0 < 1
	\end{aligned}
	\]
Because $L < 1$, the series $\ds\sum_{n=0}^\infty \dfrac{n\, 3^n}{(2n)!}$ is absolutely convergent by the Ratio Test.}



% Question 7
\newpage
\question[10] Showing all your work and fully justifying your reasoning, determine whether the following series diverges or converges. If the series converges, find its sum:
	\[
	\sum_{n=1}^\infty \left( \dfrac{1}{n + 1} - \dfrac{1}{n + 2} \right)
	\] \pspace

\vsol{\itshape Informally, we have\dots
	\[
	\begin{aligned}
	\sum_{n=1}^\infty \left( \dfrac{1}{n + 1} - \dfrac{1}{n + 2} \right)&= \left( \dfrac{1}{2} - \dfrac{1}{3} \right) + \left( \dfrac{1}{3} - \dfrac{1}{4} \right) + \left( \dfrac{1}{4} - \dfrac{1}{5} \right) + \left( \dfrac{1}{5} - \dfrac{1}{6} \right) + \cdots \\
	&= \left( \dfrac{1}{2} - \cancel{\dfrac{1}{3}} \right) + \left( \cancel{\dfrac{1}{3}} - \cancel{\dfrac{1}{4}} \right) + \left( \cancel{\dfrac{1}{4}} - \cancel{\dfrac{1}{5}} \right) + \left( \cancel{\dfrac{1}{5}} - \cancel{\dfrac{1}{6}} \right) + \cdots \\
	&= \dfrac{1}{2}
	\end{aligned}
	\]
Therefore, the series $\ds\sum_{n=0}^\infty \left( \dfrac{1}{n + 2} - \dfrac{1}{n + 3} \right)$ converges to $\frac{1}{2}$. \pspace

Formally, we know that\dots
	\[
	\sum_{n=0}^\infty \left( \dfrac{1}{n + 1} - \dfrac{1}{n + 2} \right):= \lim_{N \to \infty} \sum_{n=0}^N \left( \dfrac{1}{n + 1} - \dfrac{1}{n + 2} \right)
	\]
Observe that\dots
	\[
	\begin{aligned}
	\sum_{n=0}^N \left( \dfrac{1}{n + 1} - \dfrac{1}{n + 2} \right)&= \left( \dfrac{1}{2} - \dfrac{1}{3} \right) + \left( \dfrac{1}{3} - \dfrac{1}{4} \right) + \cdots + \left( \dfrac{1}{N + 1} - \dfrac{1}{N + 2} \right) + \left( \dfrac{1}{N + 2} - \dfrac{1}{N + 3} \right) \\
	&= \left( \dfrac{1}{2} - \cancel{\dfrac{1}{3}} \right) + \left( \cancel{\dfrac{1}{3}} - \cancel{\dfrac{1}{4}} \right) + \cdots + \left( \cancel{\dfrac{1}{N + 1}} - \cancel{\dfrac{1}{N + 2}} \right) + \left( \cancel{\dfrac{1}{N + 2}} - \dfrac{1}{N + 3} \right) \\
	&= \dfrac{1}{2} - \dfrac{1}{N + 2}
	\end{aligned}
	\]
Therefore, we have\dots
	\[
	\begin{aligned}
	\sum_{n=1}^\infty \left( \dfrac{1}{n + 1} - \dfrac{1}{n + 2} \right)&:= \lim_{N \to \infty} \sum_{n=0}^N \left( \dfrac{1}{n + 1} - \dfrac{1}{n + 2} \right) \\
	&= \lim_{N \to \infty} \left( \dfrac{1}{2} - \dfrac{1}{N + 2} \right) \\
	&= \dfrac{1}{2} - 0 \\
	&= \dfrac{1}{2}
	\end{aligned}
	\]
Therefore, the series $\ds\sum_{n=0}^\infty \left( \dfrac{1}{n + 1} - \dfrac{1}{n + 2} \right)$ converges to $\frac{1}{2}$.}



% Question 8
\newpage
\question[10] Showing all your work and fully justifying your reasoning, determine whether the following series converges or diverges:
	\[
	\sum_{n=1}^\infty \dfrac{e^n - 1}{e^n + 1}
	\] \pspace

\vsol{\itshape We have\dots
	\[
	\lim_{n \to \infty} \dfrac{e^n - 1}{e^n + 1} \stackrel{\text{\normalfont L.H.}}{=} \lim_{n \to \infty} \dfrac{e^n}{e^n}= \lim_{n \to \infty} 1= 1 \neq 0
	\] 
	\begin{center} OR \end{center}
	\[
	\lim_{n \to \infty} \dfrac{e^n - 1}{e^n + 1}= \lim_{n \to \infty} \dfrac{e^n - 1}{e^n + 1} \cdot \dfrac{1/e^n}{1/e^n}= \lim_{n \to \infty} \dfrac{1 - 1/e^n}{1 + 1/e^n}= \dfrac{1 - 0}{1 + 0}= \dfrac{1}{1}= 1 \neq 0
	\] \pspace
Therefore by the Divergence Test, the series $\ds\sum_{n=1}^\infty \dfrac{e^n - 1}{e^n + 1}$ diverges.\footnote{\itshape Note. Although one certainly does not need to apply the Integral Test in the case where the function corresponding to the summand is continuous, positive, and strictly increasing (the Divergence Test will do), one can modify the proof of the Integral Test to show that for such functions that the corresponding series diverges. In this case, one would need the fact that $\ds\int \dfrac{e^x - 1}{e^x + 1} \;dx= 2 \ln|e^x + 1| - \ln|e^x| + C= \ln\left| \dfrac{(e^x + 1)^2}{e^x} \right| + C$ and the fact that the integral $\ds\int_1^\infty \dfrac{e^x - 1}{e^x + 1} \;dx$ diverges. However, this is really beyond the scope of such a problem, so we will neither prove this variation of the Integral Test nor not make this series argument here.}
}

\end{questions}
\end{document}